\chapter{التكاملات الخطية}

في دراسة حركة جسيم على طول قوس, يكون من الملائم اعتبار القوس كصورة لدالة متجهية القيمة \( \gamma : [a, b] \to \mathbb{R}^3 \) معرفة على فترة من الخط الحقيقي وتحقيق \( \gamma(t) \) كموضع الجسيم عند الزمن \( t \). هذا المنظور يكون أيضًا ملائمًا في تحليل سلوك حقل المتجهات على طول القوس وهو الدافع الرئيسي للتعاريف التالية.

\section{ المسارات}

\begin{definition}
المسار هو دالة مستمرة \( \gamma : [a, b] \to \mathbb{R}^n \). نسمي \( \gamma(a) \) النقطة الابتدائية و \( \gamma(b) \) النقطة النهائية. تُسمى صورة المسار \( \gamma([a, b]) \) قوسًا لـ \( \gamma \). إذا كانت \( \gamma([a, b]) \subset \Omega \), نقول إن \( \gamma \) مسار في \( \Omega \).
\end{definition}

\begin{exemple}
قطعة الخط التي تصل بين نقطتين \( x \) و \( y \) في \( \mathbb{R}^n \) هي القوس \([x, y] := \gamma([0, 1])\), حيث \( \gamma : [0, 1] \to \mathbb{R}^n \) تدل على المسار
\[ \gamma(t) = x + t(y - x) = ty + (1 - t)x. \]
\end{exemple}

\begin{exemple}
لنفترض أن \( \gamma_j : [0, 2\pi] \to \mathbb{R}^2 \) معطاة بواسطة \( \gamma_j(t) := (\cos(jt), \sin(jt)) \). إذن لكل \( j \in \mathcal{Z} \setminus \{0\} \), فإن القوس \( \gamma_j([0, 2\pi]) \) هو الدائرة الوحدة \( x^2 + y^2 = 1 \) في \( \mathbb{R}^2 \). مع زيادة المعلمة \( t \) من 0 إلى 2\(\pi\), تتحرك النقطة \( \gamma_j(t) \) حول دائرة الوحدة بعدد المرات \( |j| \) (في اتجاه عقارب الساعة عندما يكون \( j \) سالبًا وعكس اتجاه عقارب الساعة عندما يكون \( j \) موجبًا).
\end{exemple}
نضع 
\[
\gamma'(t) := \lim_{h \to t, h \in [a, b]} \frac{\gamma(h) - \gamma(t)}{h - t} 
\]

إذا كانت النهاية موجودة. نلاحظ أنه إذا كان \( t \in (a, b) \), فإن \( \gamma'(t) \) موجود إذا وفقط إذا كانت \( \gamma \) دالة قابلة للتفاضل عند النقطة \( t \). في هذه الحالة, يكون \( \gamma'(t) \) هو مصفوفة عمود \( n \times 1 \) للتفاضل عند \( t \), والتي تُرى بطبيعة الحال كمتجه في \( \mathbb{R}^n \). على وجه الخصوص, يمكننا اعتبار \( \gamma' : [a, b] \to \mathbb{R}^n \), وحيثما كانت الحدود المناسبة موجودة, نكرر لاشتقاق مشتقات أعلى رتبة لـ \( \gamma \). هذا يؤدي إلى التعريفات التالية.

\begin{definition}
    
يُقال إن مسارًا \( \gamma : [a, b] \to \mathbb{R}^n \) هو دالة من الفئة \( C^q \) على $[a,b]$ إذا كانت المشتقة من الرتبة \( q \) \( \gamma^{(q)}(t) \) موجودة لكل \( t \in [a, b] \) وكانت \( \gamma^{(q)} \) مستمرة على $[a,b]$. يُقال إن الدالة \( \gamma \) دالة من قطعة إلى قطعة من الفئة \( C^q \) إذا كانت هناك تقسيمات \( a = t_1 < \cdots < t_k = b \) بحيث تكون \( \gamma|_{[t_i, t_{i+1}]} \) من الفئة \( C^q \) على \(\left[t_i,t_{i+1}\right]\) لكل $1\leq i < k-1$.
\end{definition}

\begin{definition}
    
يُقال إن مسارًا \( \gamma : [a, b] \to \mathbb{R}^n \) أملس إذا كانت \( \gamma \) دالة من الفئة \( C^1 \) وكانت \( \gamma'(t) \neq 0 \) لكل \( t \in [a, b] \).
\end{definition}

\begin{lemma}
لنفترض أن \( c \) عدد حقيقي. الدالة \( f : \mathbb{R} \to \mathbb{R} \),
\[ f(t) := \begin{cases} -c e^{-1/t^2}, t < 0, \\ 0, t = 0, \\ c e^{-1/t^2}, t > 0, \end{cases} \]
هي من الفئة \( C^\infty \) على الخط الحقيقي.
\end{lemma}

\begin{demonstration}[5]

نثبت أنه لكل \( n \in \mathcal{N} \cup \{0\} \) هناك متعددة حدود \( P_n \) بحيث
\[ f^{(n)}(t) := \begin{cases} P_n(t) e^{-1/t^2}, t < 0, \\ 0, t = 0, \\ P_n(t) e^{-1/t^2}, t > 0. \end{cases} \]

في الواقع, هذا واضح بالنسبة لـ \( n = 0 \). لتطبيق الاستقراء, نفترض أن النتيجة صحيحة لـ \( n = k \in \mathcal{N} \cup \{0\} \). لاستنتاج أن النتيجة صحيحة أيضًا لـ \( n = k + 1 \), يكفي التحقق من أن \( f^{(k+1)}(0) = 0 \). سنستخدم أن \( e^{y^2} \) يتباعد إلى ما لا نهاية أسرع من أي متعددة حدود في \( y \) عندما يتجه \( y \) إلى \( +\infty \). في الواقع, إذا كانت \( P_n(y) = a_n y^n + \cdots + a_1 y + a_0 \) مع \( a_n \neq 0 \), فإن
\[ \frac{P_n(y)}{e^{y^2}} = \frac{P_n(y)}{e^{y^2 - y}} \cdot \frac{1}{e^y}. \]

الدالة \( e^{-y^2 + y} \) بوضوح تتقارب إلى الصفر. من ناحية أخرى, بتطبيق قاعدة لوبتال \( n \) مرة, نحصل على
\[ \lim_{y \to +\infty} \frac{P_n(y)}{e^{y^2}} = \frac{a_n}{n!} \lim_{y \to +\infty} \frac{1}{e^y} = 0. \]

الآن بأخذ \( y = 1 \) عندما \( t > 0 \) أو \( y = -1 \) إذا \( t < 0 \) نحصل على
\[ f^{(k+1)}(0) = \lim_{t \to 0} \frac{f^{(k)}(t) - f^{(k)}(0)}{t} = \lim_{t \to 0} \frac{f^{(k)}(t)}{t} = 0. \]
\end{demonstration}

\begin{exemple}
لنفترض أن \( \gamma : [−1, 1] \to \mathbb{R}^2 \) معرّف بواسطة
\[ \gamma(t) := \begin{cases} \left( -e^{1/t^2}, e^{1/t^2} \right), −1 \leq t < 0, \\ (0, 0), t = 0, \\ \left( e^{1/t^2}, e^{1/t^2} \right), 0 < t \leq 1. \end{cases} \]
\end{exemple}

يتبع من التوطئة أعلاه أن \( \gamma \) هو مسار من الفئة \( C^\infty \). ومع ذلك, فإن \( \gamma \) ليس أملسًا, حيث أن \( \gamma'(0) = (0, 0) \). نلاحظ أن \( \gamma([−1, 1]) = \{(x, |x|) : x \in [−1, 1]\} \) له زاوية عند النقطة $(0,0)$. يمكن العثور على عدة أمثلة أخرى في.

مهمتنا التالية هي تعريف, وإذا أمكن, تقييم طول مسار \( \gamma : [a, b] \to \mathbb{R}^n \). الفكرة الأساسية تتكون في تقريب المسار بواسطة قطع خطية تحدد نقاط نهاياتها بواسطة تقسيم للفترة \([a, b]\). يُقال أن المسار ذو طول نهائي أنه قابل للتحديد أو ذو تباين محدود. سنبين أن كل مسار من الفئة \( C^1 \) من قطعة إلى قطعة قابل للتحديد وسنحصل على صيغة لتقييم طوله.

\begin{definition}
لنفترض أن \( \gamma : [a, b] \to \mathbb{R}^n \) مسار وأن \( P := \{a = t_1 < \cdots < t_k = b\} \) هو تقسيم للفترة \([a, b]\). القوس متعدد الأضلاع المرتبط بـ $P$ هو اتحاد القطع الخطية [\( \gamma(t_i) \), \( \gamma(t_{i+1}) \)], \(1 \leq i \leq k−1\). طول هذا القوس متعدد الأضلاع هو
\[ L(\gamma, P) := \sum_{i=1}^{k-1} \| \gamma(t_{i+1}) - \gamma(t_i) \|. \]
\end{definition}

\begin{lemma}
    لنفترض أن \( \gamma : [a, b] \to \mathbb{R}^n \) مسار وأن \( P := \{a = t_1 < \cdots < t_k = b\} \) هو تقسيم للفترة \([a, b]\). إذا كان $Q$ تقسيمًا آخر للفترة \([a, b]\) و \( P \subset Q \), فإن \( L(\gamma, P) \leq L(\gamma, Q) \).
\end{lemma}

\begin{demonstration}[5]
يمكننا الافتراض دون فقدان العمومية أن \( Q \) تم الحصول عليه من \( P \) بإضافة نقطة واحدة. لذلك, نفترض \( Q = P \cup \{s\} \), حيث \( t_j < s < t_{j+1} \). إذن
\[ L(\gamma, Q) := \sum_{i \neq j} \| \gamma(t_{i+1}) - \gamma(t_i) \| + \| \gamma(s) - \gamma(t_j) \| + \| \gamma(t_{j+1}) - \gamma(s) \|. \]

نظرًا لأن
\[ \| \gamma(t_{j+1}) - \gamma(t_j) \| \leq \| \gamma(s) - \gamma(t_j) \| + \| \gamma(t_{j+1}) - \gamma(s) \|, \]
فإن النتيجة تتبع.
\end{demonstration}

\begin{definition}
يُقال أن مسار \( \gamma \) قابل للتحديد أو ذو تباين محدود إذا
\[ \sup \{L(\gamma, P) : P تقسيم للفترة [a, b]\} < +\infty. \]

إذا كان \( \gamma \) قابلًا للتحديد, فسنشير إلى هذا الأعلى بطول \( \gamma \), وسنرمز له بـ \( L(\gamma) \).
\end{definition}

لنفرض أن \( \gamma : [a, b] \to \mathbb{R}^n \) مسار وأن \( a < c < b \). إذن \( \gamma \) قابل للتحديد إذا وفقط إذا كانت \( \gamma|_{[a, c]} \) و \( \gamma|_{[c, b]} \) قابلتين للتحديد. وعلاوة على ذلك, في هذه الحالة,
\[ L(\gamma) = L(\gamma|_{[a, c]}) + L(\gamma|_{[c, b]}). \]
اثبت ذلك.

\begin{definition}
    
معيار تقسيم \( P := \{a = t_1 < \cdots < t_k = b\} \) هو طول أكبر فترة فرعية معرّفة بذلك التقسيم, أي
\[ \|P\| = \max \{|t_{i+1} - t_i| : i = 1, \ldots, k - 1\}. \]
\end{definition}

سنوضح أن أي مسار من الفئة \( C^1 \) قابل للتحديد, ولكن قبل أن نفعل ذلك, نحتاج إلى تذكير القارئ بمفهوم التراص الطوبولوجي وبعض الخصائص التي تتمتع بها المجموعات المدمجة.

\begin{definition}
تُسمى مجموعة فرعية \( K \) من \( \mathbb{R}^n \) مدمجة إذا كان لكل عائلة \( F \) من المجموعات المفتوحة في \( \mathbb{R}^n \) التي تغطي \( K \), بمعنى أن
\[ K \subset \bigcup_{G \in F} G, \]
توجد عائلة فرعية منتهية \( G_1, \ldots, G_m \) في \( F \) بحيث أن \( K \subset \bigcup_{j=1}^{m} G_j \).
\end{definition}

تُسمى مجموعة فرعية \( M \) من \( \mathbb{R}^n \) محدودة إذا كانت محتواة في كرة مفتوحة مركزها عند الأصل. إذا كانت \( K \) مجموعة مدمجة, فإنها محدودة. هذا تمرين سهل, لكن يمكن قول المزيد. يمكن العثور على دليل لتوصيفات المجموعات المدمجة في \( \mathbb{R}^n \) في أي كتاب تقريبًا عن حساب التفاضل في عدة متغيرات.

\begin{theoreme}
لنفترض أن \( K \) مجموعة فرعية من \( \mathbb{R}^n \). الشروط التالية مكافئة:
1. \( K \) مدمجة.
2. لكل تسلسل \( (x_j) \subset K \) يوجد تسلسل فرعي \( (x_{j_k}) \) متقارب إلى نقطة \( x_0 \in K \).
3. (نظرية هاين-بوريل-ليبيج) \( K \) محدودة ومغلقة في \( \mathbb{R}^n \).
\end{theoreme}

نحتاج أيضًا إلى مفهوم الاستمرارية المنتظمة.

\begin{definition}
لنفترض أن \( M \) مجموعة فرعية من \( \mathbb{R}^n \). تُسمى دالة \( f : M \subset \mathbb{R}^n \to \mathbb{R}^m \) مستمرة بانتظام على \( M \) إذا كان لكل \( \epsilon > 0 \), يوجد \( \delta > 0 \) بحيث لكل \( x \) و \( y \) في \( M \) مع \( \|x - y\| < \delta \), لدينا
\[ \|f(x) - f(y)\| < \epsilon. \]
\end{definition}

الاستمرارية المنتظمة تعني بالطبع الاستمرارية, لكنها في الواقع خاصية أقوى. ومع ذلك, فإن كلا المفهومين يتطابقان إذا كانت المجموعة \( M \) مجموعة مدمجة, وهي نتيجة نذكرها هنا دون دليل.

\begin{theoreme}[(نظرية هاين-كانتور)]
كل دالة مستمرة \( f : K \subset \mathbb{R}^n \to \mathbb{R}^m \) على مجموعة مدمجة \( K \) مستمرة بانتظام على \( K \).
\end{theoreme}

\begin{theoreme}
لنفترض أن \( \gamma : [a, b] \to \mathbb{R}^n \) مسار من الفئة \( C^1 \). إذن \( \gamma \) قابل للتحديد و
\[
L(\gamma) = \int_a^b \|\gamma'(t)\| \, dt.
\]
\end{theoreme}

\begin{demonstration}
الدليل: نعرّف \(F : [a, b]^n \to \mathbb{R}\) بواسطة
\[ F(s_1, \ldots, s_n) := \sum_{j=1}^n |\dot{\alpha}_j(s_j)|^2. \]
ثم
\[ F(t, \ldots, t) = ||\dot{\alpha}(t)||. \]
علاوة على ذلك, بالنظر إلى تقسيم
\[ P := \{a = t_1 < \cdots < t_k = b\}, \]
يمكننا تطبيق نظرية القيمة المتوسطة لاستنتاج أنه لكل \(1 \leq i \leq k - 1\) و \(1 \leq j \leq n\), يوجد
\[ s_{ji} \in [t_i, t_{i+1}] \]
بحيث
\[ L(\alpha, P) = \sum_{i=1}^{k-1} \|\alpha(t_{i+1}) - \alpha(t_i)\| = \sum_{i=1}^{k-1} \sqrt{\sum_{j=1}^n (\alpha_j(t_{i+1}) - \alpha_j(t_i))^2} = \sum_{i=1}^{k-1} F(s_{1i}, \ldots, s_{ni})(t_{i+1} - t_i). \]
نظرًا لأن \(F\) دالة مستمرة على المجموعة المضغوطة \([a, b]^n\), فإنه يتبع من نظرية هاينه-كانتور أن \(F\) في الواقع مستمرة بشكل منتظم على \([a, b]^n\). بمعنى, لكل \(\epsilon > 0\) يوجد \(\delta > 0\) بحيث أن لكل \(x, y \in [a, b]^n\) و \(\|x - y\| < \delta\) ينتج
\[ |F(x) - F(y)| < \epsilon. \]
لنفترض الآن أن التقسيم السابق \(P\) يحقق \(\|P\| < \delta\). إذن
\[ \|\alpha(t, \ldots, t) - (s_{1i}, \ldots, s_{ni})\| < \delta \]
كلما
\[ t \in [t_i, t_{i+1}]. \]
وبالتالي
\[ L(\alpha, P) - \int_a^b \|\dot{\alpha}(t)\| dt = \sum_{i=1}^{k-1} \int_{t_i}^{t_{i+1}} |F(s_{1i}, \ldots, s_{ni}) - F(t, \ldots, t)| dt \leq \epsilon \sum_{i=1}^{k-1} (t_{i+1} - t_i) = \epsilon (b - a). \]
لإكمال الدليل, نثبت تقسيم \(P_0\) بمعيار أقل من \(\delta\). بالنسبة لأي تقسيم \(P\) للفترة \([a, b]\) لدينا:
\[ L(\alpha, P) \leq L(\alpha, P \cup P_0) \leq \epsilon (b - a) + \int_a^b \|\dot{\alpha}(t)\| dt. \]
الذي يثبت أن \(\alpha\) قابل للقياس والطول
\[ L(\alpha) \leq \epsilon (b - a) + \int_a^b \|\dot{\alpha}(t)\| dt. \]
 \breakbox
من ناحية أخرى,
\[ L(\alpha) \geq L(\alpha, P_0) \geq \int_a^b \|\dot{\alpha}(t)\| dt - \epsilon (b - a). \]
عند أخذ النهاية حيث \(\epsilon\) تقترب من الصفر, نصل إلى النتيجة المطلوبة.
\end{demonstration}

\begin{lemma}
لنفترض أن \( \gamma : [a, b] \to \mathbb{R}^n \) مسار من الفئة \( C^1 \) من قطعة إلى قطعة. إذن \( \gamma \) قابل للتحديد و
\[ L(\gamma) = \int_a^b \|\gamma'(t)\| \, dt. \]
\end{lemma}
\begin{demonstration}

لنفترض أن \(\{a = t_1 < t_2 < \cdots < t_k = b\}\) هو تقسيم للفترة $[a, b]$ بحيث أن \( \gamma_j := \gamma|_{[t_j, t_{j+1}]} \) هو من الفئة \( C^1 \) لكل \( j = 1, \cdots, k - 1 \). وفقًا لنظرية 2.1.3, كل \( \gamma_j \) هو مسار قابل للتحديد على $[t_j, t_{j+1}]$ و
\[ L(\gamma_j) = \int_{t_j}^{t_{j+1}} \|\gamma_j'(t)\| \, dt. \]

نتيجة لذلك, فإن \( \gamma \) أيضًا قابل للتحديد و
\[ L(\gamma) = \sum_{j=1}^{k-1} L(\gamma_j) = \sum_{j=1}^{k-1} \int_{t_j}^{t_{j+1}} \|\gamma_j'(t)\| \, dt. \]

الدالة \( t \to \|\gamma'(t)\| \) معرفة جيدًا عند كل نقطة من الفترة $[a, b]$ باستثناء مجموعة منتهية. علاوة على ذلك, فإن تقييدها لكل فترة $[t_j, t_{j+1}]$ مستمر, ومن ثم قابل للتكامل ريمان على $[t_j, t_{j+1}]$ لكل \( j = 1, \cdots, k - 1 \). يتبع ذلك أن الدالة قابلة للتكامل ريمان على $[a, b]$. يمكننا بالتالي الاستنتاج, من خصائص الدوال القابلة للتكامل ريمان, أن
\[ L(\gamma) = \int_a^b \|\gamma'(t)\| \, dt. \]
\end{demonstration}

لنفرض أن \( \gamma : [0, 1] \to \mathbb{R}^2 \) معطاة بواسطة \( \gamma(0) := (0, 0) \) و \( \gamma(t) := (t, t \cos(\pi/t)) \) عندما \( 0 < t \leq 1 \). نترك للقارئ إثبات أن \( \gamma \) مسار مستمر ولكنه غير قابل للتحديد. على الرغم من أننا نركز على المسارات من الفئة \( C^1 \) من قطعة إلى قطعة في النص, فقد وجدنا أنه من الطبيعي التعامل مع الفئة الأكثر عمومية من المسارات القابلة للتحديد في معالجتنا لطول المسار. التعميم مفيد أيضًا في القسم التالي, حيث نعتبر العمل المنجز بواسطة حقل المتجهات.
%###################
\section{تكامل الحقول المتجهة}
كان الدافع الاساسي للتكامل الخطي هو المشاكل المتعلقة بحركة السوائل والمجالات الكهرومغناطيسية أو غيرها من مجالات القوة. لنفترض, على سبيل المثال, أن \( \gamma : [a, b] \to \mathbb{R}^3 \) هو مسار أملس موجود في المجموعة المفتوحة \( U \subset \mathbb{R}^3 \) وأن هناك مجال قوة \( F : U \to \mathbb{R}^3 \). نريد تقييم العمل المنجز بواسطة القوة على جسم يتحرك على طول القوس \( \gamma([a, b]) \) من \( \gamma(a) \) إلى \( \gamma(b) \). سيتعين علينا أن نأخذ في الاعتبار مبدأين أساسيين:
1. يعتمد العمل فقط على مركبة القوة التي تعمل في نفس اتجاه حركة الجسم (أي الاتجاه المماسي للمسار عند كل نقطة).
2. العمل المنجز بواسطة مجال ثابت \( F_0 \) لتحريك الجسم عبر قطعة خطية, في نفس اتجاه \( F_0 \), هو حاصل ضرب \( \|F_0\| \) وطول تلك القطعة.

نتذكر أن
\[ T(t) := \frac{\gamma'(t)}{\|\gamma'(t)\|} \]
هو متجه وحدة مماسي للمسار عند \( \gamma(t) \), وبالتالي فإن مركبة القوة \( F \) التي تعمل في الاتجاه المماسي للمسار عند \( \gamma(t_j) \) هي
\[ \langle F(\gamma(t_j)), T(t_j) \rangle. \]

إذا اعتبرنا فترة قصيرة \( [t_j, t_{j+1}] \), فإن طول \( \gamma|_{[t_j, t_{j+1}]} \) يُقَدّر بواسطة
\[ \|\gamma'(t_j)\| \cdot (t_{j+1} - t_j). \]

ومن ثم, باستخدام المبادئ الأساسية المذكورة أعلاه على تقسيم دقيق جدًا \( P := \{a = t_1 < \cdots < t_k = b\} \) للفترة, نرى أن تقريبًا جيدًا للعمل المنجز في تحريك الجسيم على طول \( \gamma \) هو
\[ \sum_{j=1}^{k-1} \langle F(\gamma(t_j)), \gamma'(t_j) \rangle \cdot (t_{j+1} - t_j). \]

النتيجة التالية تخبرنا بقيمة الحد لهذا المقدار عندما نأخذ تقسيمات أدق وأدق (الشكل 2.1).

\begin{figure}
    \centering
    \includegraphics[width=0.6\linewidth]{vector_field_along_path.pdf}
    \caption{حقل متجه على طول مسار}
    \label{fig:enter-label}
\end{figure}
\begin{lemma}

لنفترض أن \( P := \{a = t_1 < t_2 < \cdots < t_k = b\} \) هو تقسيم للفترة [a, b]. لأي دالة مستمرة \( f : [a, b] \to \mathbb{R} \) ولكل اختيار \( u_j \in [t_j, t_{j+1}] \), لدينا:
\[ \lim_{ \|P\| \to 0 } \sum (t_{j+1} - t_j) \cdot f(u_j) = \int_a^b f(t) \, dt. \]

\end{lemma}

\begin{demonstration}

وفقًا لنظرية هاين-كانتور, فإن \( f \) مستمرة بانتظام, مما يعني أنه لكل \( \epsilon > 0 \) يمكن العثور على \( \delta > 0 \) بحيث أن \( |f(s) - f(t)| \leq \epsilon \) عندما يكون \( s \), \( t \in [a, b] \) و \( |s - t| \leq \delta \). لننظر الآن في تقسيم \( P = \{a = t_1 < t_2 < \cdots < t_k = b\} \) بحيث \( 0 < \|P\| < \delta \). إذن:

\[ \left| \sum_{j=1}^{k-1} (t_{j+1} - t_j) \cdot f(u_j) - \int_a^b f(t) \, dt \right| \leq \epsilon (b - a). \]

نظرًا لأن التقسيمات تصبح أدق وأدق, فإن مجموع ريمان يقترب من التكامل المحدد, وبالتالي:

\[ \lim_{\|P\| \to 0} \sum_{j=1}^{k-1} (t_{j+1} - t_j) \cdot f(u_j) = \int_a^b f(t) \, dt. \]
\begin{eqnarray*}
\left| \sum_{j=1}^{k-1} (t_{j+1} - t_j) \cdot f(u_j) - \int_a^b f(t) \, dt \right|
&=& \left| \sum_{j=1}^{k-1} (t_{j+1} - t_j) \cdot f(u_j) - \sum_{j=1}^{k-1} \int_{t_j}^{t_{j+1}} f(t) \, dt \right| \\
&\leq& \sum_{j=1}^{k-1} \left| (t_{j+1} - t_j) \cdot f(u_j) - \int_{t_j}^{t_{j+1}} f(t) \, dt \right| \\
 &\leq& \sum_{j=1}^{k-1} \int_{t_j}^{t_{j+1}} \left| f(u_j) - f(t) \right| dt\\
 \leq \epsilon \sum_{j=1}^{k-1} (t_{j+1} - t_j) = \epsilon (b - a).
 \end{eqnarray*}
وبما أن \( \epsilon \) صغيرة بشكل تعسفي, فإن النتيجة تتبع.

\end{demonstration}

تُظهر التوطئة والمناقشة السابقة لها أن العمل المنجز بواسطة مجال القوة \( F \) على جسيم أثناء حركته على طول \( \gamma \) هو
\[ \int_a^b \langle F(\gamma(t)), \gamma'(t) \rangle \, dt. \]

هذا النوع من التكامل, المعروف بالتكامل الخطي, يظهر كثيرًا في الفصول التالية, ولذلك نقدم تعريفًا رسميًا.

\begin{definition}
لنفترض أن \( F : U \subset \mathbb{R}^n \to \mathbb{R}^n \) هو حقل متجهات مستمر وأن \( \gamma \) هو مسار من الفئة \( C^1 \) من قطعة إلى قطعة في \( U \). يُعطى التكامل الخطي لـ \( F \) على طول \( \gamma \) بواسطة
\[ 
\int_\gamma F := \int_a^b \langle F(\gamma(t)), \gamma'(t) \rangle \, dt. 
\]
\end{definition}


لنعتبر دالة متجهة \(F : U \subset \mathbb{R}^n \to \mathbb{R}^n\), حيث \(U\) هي مجموعة مفتوحة. إذا كانت \(\alpha : [a, b] \to U\) مسارًا منتظمًا بحيث أن مشتقها \(\dot{\alpha}\) موجود ومستمر على \([a, b]\), فإننا نعرف التكامل الخطي لـ\(F\) على \(\alpha\) كما يلي:
\[ \int_\alpha F \cdot d\alpha = \int_a^b F(\alpha(t)) \cdot \dot{\alpha}(t) \, dt. \]

الذي يمثل الشغل الذي تقوم به القوة \(F\) في نقل جسم على طول المسار \(\alpha\).


\section{ تغييرات المعلمة}

في بعض الأحيان, يكون من المفيد إعادة صياغة تكامل خطي باستخدام تغيير في المعلمة. لنفرض أن \(\beta : [c, d] \to [a, b]\) دالة قابلة للتفاضل من الدرجة الأولى وتحقق \(\beta(c) = a\) و \(\beta(d) = b\). إذا كان لدينا مسار منتظم \(\alpha : [a, b] \to \mathbb{R}^n\) فإننا نستطيع تعريف مسار جديد \(\gamma : [c, d] \to \mathbb{R}^n\) بواسطة \(\gamma(t) = \alpha(\beta(t))\). عندها يكون:
\[ \int_\gamma F \cdot d\gamma = \int_c^d F(\gamma(t)) \cdot \dot{\gamma}(t) \, dt = \int_c^d F(\alpha(\beta(t))) \cdot (\alpha'(\beta(t)) \beta'(t)) \, dt. \]

يعتمد التكامل الخطي \( \int_\gamma F \) على حقل المتجهات \( F \) وكذلك على المسار \( \gamma \). في هذا القسم, نخطط لتحليل ما يحدث عند استبدال \( \gamma \) بمسار آخر له نفس الأثر.

\begin{definition}
    
لنفترض أن \( \alpha : [a, b] \to \mathbb{R}^n \) و \( \beta : [c, d] \to \mathbb{R}^n \) هما مساران. نقول أن \( \alpha \) و \( \beta \) متكافئان, ونكتب \( \alpha \sim \beta \), إذا كان هناك دالة \( \varphi : [a, b] \to [c, d] \) من الفئة \( C^1 \), بحيث أن \( \varphi([a, b]) = [c, d] \), \( \varphi'(t) > 0 \) لكل \( t \in [a, b] \) و \( \alpha = \beta \circ \varphi \) (انظر الشكل 2.2).
\end{definition}

بواسطة نظرية القيمة المتوسطة, فإن الشروط على \( \varphi \) في هذا التعريف تضمن أن \( \varphi \) تزايدية بدقة, ومن ثم تكون شاملة, وأن \( c = \varphi(a) \) و \( d = \varphi(b) \). يتبع ذلك أن \( \alpha \sim \beta \) يعني \( \beta \sim \alpha \). في الواقع, \( \beta = \alpha \circ \varphi^{-1} \), و \( \varphi^{-1} \) لها الخصائص اللازمة للتكافؤ.

عادةً ما يشار إلى النتيجة التالية باسم قاعدة السلسلة أو نظرية الدالة المركبة. 

\begin{theoreme}
لنفترض أن \( U \) و \( V \) مجموعات مفتوحة من \( \mathbb{R}^n \) و \( \mathbb{R}^m \) على التوالي. إذا كانت الدالتان \( F : U \to V \) و \( G : V \to \mathbb{R}^p \) قابلتين للتفاضل عند \( a \in U \) و \( F(a) \in V \) على التوالي, فإن تركيبتهما \( H = G \circ F \) قابلة للتفاضل عند \( a \), و
\[ dH(a) = dG(F(a)) \circ dF(a), \]
أو, بعبارات مصفوفاتهما المرتبطة,
\[ H'(a) = G'(F(a)) \cdot F'(a). \]
علاوة على ذلك, إذا كانت \( F \) و \( G \) من الفئة \( C^q \, 1 \leq q \leq \infty\) على مجالات تعريفهما الخاصة, فإن \( H \) أيضًا من الفئة \( C^q \) على \( U \).
\end{theoreme}

نلاحظ أن فرضية قاعدة السلسلة تتضمن دوال معرفة على مجموعات مفتوحة. ومع ذلك, عندما تكون الدالة الأولى معرفة على فترة مغلقة $[a, b]$, فإن نوعًا آخر من النتيجة يكون صحيحًا أيضًا.

\begin{figure}
    \centering
    \includegraphics[width=0.75\linewidth]{equivalent path.pdf}
    \caption{مسارات متكافئة}
    \label{fig:enter-label}
\end{figure}

\begin{lemma}
لنفترض أن \( V \) مجموعة مفتوحة من \( \mathbb{R}^m \), و \( \gamma : [a, b] \to V \) مسار له مشتق عند كل نقطة من $[a, b]$, ولنفرض أن \( G : V \to \mathbb{R}^m \) قابل للتفاضل عند كل نقطة من \( \gamma([a, b]) \). إذن \( G \circ \gamma : [a, b] \to \mathbb{R}^m \) له مشتق عند كل نقطة من $[a, b]$ و
\[ (G \circ \gamma)'(t) = G'(\gamma(t)) \cdot \gamma'(t), \]
لكل \( t \in [a, b] \). هنا, تُفهم المشتقات عند \( t = a \) و \( t = b \) كمشتقات من اليمين ومن اليسار على التوالي. علاوة على ذلك, إذا كانت \( \gamma \) من الفئة \( C^q \) على $[a, b]$ و \( G \) من الفئة \( C^q \) على المجموعة المفتوحة \( V \), فإن \( G \circ \gamma \) تكون أيضًا من الفئة \( C^q \) على $[a, b]$.
\end{lemma}

\begin{demonstration}
    
نكتب \( \gamma(t) = (\gamma_1(t), \ldots, \gamma_m(t)) \), حيث \( \gamma_j : [a, b] \to \mathbb{R} \) لها مشتق على $[a, b]$ لكل \( j = 1, \codes, n \) (أو من الفئة \( C^q \) على $[a, b]$ على التوالي).

من المعروف أن \( \gamma_j \) يمكن تمديدها إلى \( \tilde{\gamma}_j : \mathbb{R} \to \mathbb{R} \) مع مشتق عند كل نقطة من \( \mathbb{R} \) (أو من الفئة \( C^q \) على \( \mathbb{R} \) على التوالي). نأخذ \( \tilde{\gamma} = (\tilde{\gamma}_1, \codes, \tilde{\gamma}_m) : \mathbb{R} \to \mathbb{R}^m \). الآن نعرّف \( U = (\tilde{\gamma})^{-1}(V) \). نظرًا لأن \( \tilde{\gamma} \) مستمرة على \( \mathbb{R} \), فإن \( U \) هي مجموعة مفتوحة من \( \mathbb{R} \) تحتوي على [a, b]. علاوة على ذلك, لدينا \( \tilde{\gamma} : U \to V \) و \( G : V \to \mathbb{R}^m \) مع \( \tilde{\gamma} \) و \( G \) قابلتين للتفاضل (على التوالي من الفئة \( C^q \)) على مجالات تعريفهما الخاصة. من قاعدة السلسلة نحصل على النتيجة.
\end{demonstration}

نفس الحجة تثبت بوضوح التباين التالي لقاعدة السلسلة.

\begin{lemma}
    
لنفترض أن \( \alpha : [a, b] \to \mathbb{R} \) قابلة للتفاضل عند كل نقطة من $[a, b]$ وأن \( \beta : [c, d] \to \mathbb{R}^m \) مسار له مشتق عند كل نقطة من $[c, d]$ مع \( \alpha([a, b]) \subset [c, d] \). إذن \( \beta \circ \alpha : [a, b] \to \mathbb{R}^m \) لها مشتق عند كل نقطة من $[a, b]$ و
\[ (\beta \circ \alpha)'(t) = \beta'(\alpha(t)) \cdot \alpha'(t), \]
لكل \( t \in [a, b] \). علاوة على ذلك, إذا كانت \( \alpha \) من الفئة \( C^q \) على $[a, b]$ و \( \beta \) من الفئة \( C^q \) على $[c, d]$, فإن \( \beta \circ \alpha \) تكون أيضًا من الفئة \( C^q \) على $[a, b]$.
\end{lemma}

\begin{lemma}
إذا كان \( \alpha \) و \( \beta \) مساران متكافئان وكان أحدهما, على سبيل المثال \( \beta \), من الفئة \( C^1 \) من قطعة إلى قطعة, فإن \( \alpha \) أيضًا من الفئة \( C^1 \) من قطعة إلى قطعة و \( L(\alpha) = L(\beta) \).
\end{lemma}

\begin{demonstration}
نكتب \( \alpha = \beta \circ \varphi \), حيث أن \( \varphi \) كما في التعريف السابق ولنفرض أن \( Q = \{c = u_1 < \cdots < u_k = d\} \) تقسيم للفترة $[c, d]$ بحيث أن \( \beta|_{[u_j, u_{j+1}]} \) من الفئة \( C^1 \) على $[u_j, u_{j+1}]$.

إذا أخذنا \( t_j := \varphi^{-1}(u_j) \), فإن \( P = \{a = t_1 < \cdots < t_k = b\} \) هو تقسيم للفترة $[a, b]$.

نظرًا لأن \( \alpha|_{[t_j, t_{j+1}]} = (\beta \circ \varphi)|_{[t_j, t_{j+1}]} = (\beta|_{[u_j, u_{j+1}]}) \circ \varphi|_{[t_j, t_{j+1}]} \) هو تركيب دالتين من الفئة \( C^1 \) (على الفترات المغلقة), فإنه يكون نفسه من الفئة \( C^1 \) على $[t_j, t_{j+1}]$, وبالتالي يكون \( \alpha \) من الفئة \( C^1 \) من قطعة إلى قطعة. الآن,
\[ L(\beta) = \int_{c=\varphi(a)}^{d=\varphi(b)} \|\beta'(u)\| \, du = \sum_{j=1}^{k-1} \int_{u_j}^{u_{j+1}} \|\beta'(u)\| \, du. \]

لكل \( 1 \leq j \leq k - 1 \) يمكننا تطبيق تغيير المتغير \( u = \varphi(t) \) للحصول على

\begin{eqnarray*}
    \int_{u_j}^{u_{j+1}} \| \beta'(u) \| du &=&  \int_{t_j}^{t_{j+1}} \| \beta'(\varphi(t)) \|\varphi'(t) dt
    \\
    &=&  \int_{t_j}^{t_{j+1}} \| \beta'(\varphi(t)) \varphi'(t)\| dt = \int_{t_j}^{t_{j+1}} \| (\beta \circ \varphi)'\| dt
    \\
    \int_{t_j}^{t_{j+1}} \| \alpha'(t)\| dt
\end{eqnarray*}

بالجمع لجميع القيم \( 1 \leq j \leq k - 1 \), نستنتج أن
\[ 
L(\beta) = \int_a^b \|\alpha'(t)\| \, dt = L(\alpha). 
\]

\end{demonstration}

\begin{lemma}
لنفترض أن \( \omega \) هي 1-form مستمرة على \( U \), وأن \( \alpha \) و \( \beta \) مساران من الفئة \( C^1 \) من قطعة إلى قطعة في \( U \) بحيث \( \alpha \sim \beta \). إذن
\[ \int_\alpha \omega = \int_\beta \omega. \]
\end{lemma}

\begin{demonstration}
    
دعنا نفترض أولاً أن \( \alpha \) و \( \beta \) مساران من الفئة \( C^1 \). ثم, باستخدام نفس التدوين كما في التمهيدية السابقة, لدينا

\begin{eqnarray*}
    \int_{\beta} \omega &=&  \int_{\varphi(a)}^{\varphi(b)} \omega(\beta(u)) (\beta'(u)) du
    \\
    &=&  \int_{a}^b  \omega(\beta(\varphi(t))) (\beta'(\varphi(t))) \varphi'(t) dt
    \\
    \int_{a}^b  \omega(\beta(\varphi(t))) (\beta \circ \varphi(t)) dt
    \\
    \int_{a}^b  \omega(\alpha(t)) (\alpha'(t)) dt = \int_{\alpha} \omega.
\end{eqnarray*}

Here is the translation of the provided text into Arabic:

---

مرة أخرى نستخدم نظرية قاعدة السلسلة لامتدادات \( \beta \) و \( \phi \) إلى خط الأعداد الحقيقي بأكمله. في الحالة العامة, نعتبر تقسيم
\[ P := \{a = t_1 < t_2 < \cdots < t_k = b\} \]
للفترة $[a, b]$ حيث تكون \( \beta \) من الفئة \( C^1 \) على كل فترة فرعية $[u_j, u_{j+1}] = [\phi(t_j), \phi(t_{j+1})]$. إذن \( \alpha = \beta \circ \phi \) هي أيضًا من الفئة \( C^1 \) على كل فترة فرعية $[t_j, t_{j+1}]$. ولذلك لدينا من أعلاه أن
\[ \int_{\beta_j} \omega = \int_{\alpha_j} \omega, \]
حيث أن \( \beta_j = \beta|_{[u_j, u_{j+1}]} \) و \( \alpha_j = \alpha|_{[t_j, t_{j+1}]} \). بجمع هذه الهوية لجميع القيم من \( j \) من 1 إلى \( k - 1 \), نحصل على
\[ \int_\beta \omega = \int_\alpha \omega. \]
\end{demonstration}

\begin{figure}
    \centering
    \includegraphics[width=0.5\linewidth]{opposit paths.pdf}
    \caption{$\gamma$ و  $-\gamma$ مسارين متعاكسين}
    \label{fig:enter-label}
\end{figure}

\begin{definition}
لنفترض أن \( \gamma : [a, b] \to \mathbb{R}^n \) هو مسار من الفئة \( C^1 \) من قطعة إلى قطعة. يتم تعريف المسار المعاكس (الشكل 2.3) بواسطة \( (-\gamma) : [−b, −a] \to \mathbb{R}^n \), بحيث \( (-\gamma)(t) := \gamma(−t) \).
\end{definition}

المسار المعاكس \( -\gamma \) يتحرك على طول \( \gamma([a, b]) \) في الاتجاه المعاكس لذلك الذي يعطيه المسار \( \gamma \). النقطة الابتدائية للمسار \( -\gamma \) هي النقطة النهائية للمسار \( \gamma \) والعكس بالعكس. سيلاحظ القارئ الفرق بين \( (-\gamma)(t) \) و \( -(\gamma(t)) \).

\begin{definition}
لنفترض أن \( \alpha : [a, b] \to \mathbb{R}^n \) و \( \beta : [c, d] \to \mathbb{R}^n \) هما مساران من الفئة \( C^1 \) من قطعة إلى قطعة بحيث \( \alpha(b) = \beta(c) \). نعني باتحاد المسارات \( \alpha \) و \( \beta \), المشار إليه بـ \( \alpha \cup \beta \), أي مسار من الفئة \( C^1 \) من قطعة إلى قطعة \( \gamma : [e, f] \to \mathbb{R}^n \) مع الخاصية أنه لبعض \( e < r < f \),
\[ \gamma|_{[e, r]} \sim \alpha \] و \[ \gamma|_{[r, f]} \sim \beta. \]
\end{definition}

من الواضح أن أثر \( \alpha \cup \beta \) هو \( \alpha([a, b]) \cup \beta([c, d]) \) و \( \alpha \cup \beta \) يتكون من تتبع \( \alpha \) أولاً ثم \( \beta \). مثل هذا الاتحاد من المسارات موجود دائمًا, ومثال ملموس يُعطى بواسطة \( \gamma : [0, 1] \to \mathbb{R}^n \), حيث
\[ \gamma(t) = \begin{cases} 
\alpha\left(\frac{2t(b - a) + a}{2}\right), & \text{إذا كان } 0 \leq t \leq \frac{1}{2}, \\
\beta\left((2t - 1)(d - c) + c\right), & \text{إذا كان } \frac{1}{2} \leq t \leq 1.
\end{cases} \]

**الاقتراح 2.4.3.**
لنفترض أن \( \omega \) هي 1-form مستمرة على مجموعة مفتوحة \( U \) من \( \mathbb{R}^n \) وأن \( \alpha \) و \( \beta \) و \( \gamma \) هي ثلاثة مسارات من الفئة \( C^1 \) من قطعة إلى قطعة في \( U \) مع نقطة النهاية للمسار \( \alpha \) تتطابق مع النقطة الابتدائية للمسار \( \beta \). إذن:
1. \( \int_{-\gamma} \omega = -\int_\gamma \omega \).
2. \( \int_{\alpha \cup \beta} \omega = \int_\alpha \omega + \int_\beta \omega \).

**الدليل.**
بما أن \( (-\gamma)'(t) = -\gamma'(-t) \), فإن التبديل \( u = -t \) يعطي
\[ \int_{-\gamma} \omega = \int_{-a}^{-b} \omega(\gamma(-t)) \cdot (-\gamma'(-t)) \, dt = \int_{b}^{a} -\omega(\gamma(u)) \cdot \gamma'(u) \, du = -\int_\gamma \omega. \]

أيضًا,
\[ \int_{\alpha \cup \beta} \omega = \int_e^r \omega(\gamma(t)) \cdot \gamma'(t) \, dt + \int_r^f \omega(\gamma(t)) \cdot \gamma'(t) \, dt = \int_\alpha \omega + \int_\beta \omega. \]


\section{ الحقول المحافظة: الأشكال التفاضلية التامة}


نبدأ هذا القسم بمثال على حقل متجهات يكون فيه التكامل الخطي على مسارين مختلفين من (0, 0) إلى (1, 1) هو نفسه.

\begin{exemple}
لنفترض أن \( F : \math{R}^2 \to \math{R}^2 \) هو حقل المتجهات المحدد بواسطة \( F(x, y) = (x, y) \) ولنعتبر المسارين (الشكل 2.4)
\[ \gamma_1 : [0, 1] \to \math{R}^2 \] و\[ \gamma_2 : [0, 1] \to \math{R}^2 \] المعرفين بواسطة \[ \gamma_1(t) = (t, t) \] و\[ \gamma_2(t) = (t, t^2) \]. إذن
\[ \int_{\gamma_1} F = \int_0^1 \langle (t, t), (1, 1) \rangle \, dt = \int_0^1 (t + t) \, dt = 1. \]
وأيضًا
\[ \int_{\gamma_2} F = \int_0^1 \langle (t, t^2), (1, 2t) \rangle \, dt = \int_0^1 (t + 2t^3) \, dt = 1. \]
\end{exemple}

\begin{figure}
    \centering
    \includegraphics[width=0.5\linewidth]{path example.pdf}
    \caption{}
    \label{fig:enter-label}
\end{figure}

حقيقة أن التكاملين الخطيين هنا يأخذان نفس القيمة ليست مفاجئة إذا تذكرنا أن التكامل الخطي يمثل العمل المنجز بواسطة حقل قوة في تحريك جسيم على طول مسار, ومن المعروف في الفيزياء أنه تحت تأثير حقل الجاذبية, العمل المنجز لنقل جسم بين نقطتين مختلفتين يعتمد فقط على فرق طاقات الجهد عند هاتين النقطتين, وبالتالي يكون مستقلًا عن المسار المتخذ (انظر المثال 2.5.4). مثالنا التالي, حقل متجهات يكون فيه التكامل الخطي على مسارين مختلفين من (0, 0) إلى (1, 1) مختلفًا, يوضح أن هذا السلوك ليس عالميًا.

\begin{exemple}
    
**المثال 2.5.2.** 
لنفترض أن \( \gamma_1 \) و \( \gamma_2 \) هما المساران المعطيان في المثال السابق ولكن دعنا نعتبر بدلاً من ذلك حقل المتجهات
\[ F(x, y) = (-y + 3, x - 1). \]
إذن
\[ \int_{\gamma_1} F = \int_0^1 \langle (-t + 3, t - 1), (1, 1) \rangle \, dt = \int_0^1 (-t + 3 + t - 1) \, dt = \int_0^1 (2) \, dt = 2. \]

وأيضًا
\begin{eqnarray*}
     \int_{\gamma_2} F &=& \int_0^1 \langle (-t^2 + 3, t - 1), (1, 2t) \rangle \,dt 
     \\
     &=& \int_0^1 (-t^2 + 3 + 2t(t - 1)) \, dt = \int_0^1 (-t^2 + 3 + 2t^2 - 2t) \, dt 
     \\
     &=& \int_0^1 (t^2 - 2t + 3) \, dt.
\end{eqnarray*}
لحساب هذا التكامل, نلاحظ أن:
\[ \int_0^1 (t^2 - 2t + 3) \, dt = \int_0^1 t^2 \, dt - 2 \int_0^1 t \, dt + \int_0^1 3 \, dt. \]

نحسب هذه التكاملات على التوالي:
\[ \int_0^1 t^2 \, dt = \left[ \frac{t^3}{3} \right]_0^1 = \frac{1}{3}, \]
\[ 2 \int_0^1 t \, dt = 2 \left[ \frac{t^2}{2} \right]_0^1 = 1, \]
\[ \int_0^1 3 \, dt = 3 \left[ t \right]_0^1 = 3. \]

وبالتالي:
\[ \int_0^1 (t^2 - 2t + 3) \, dt = \frac{1}{3} - 1 + 3 = \frac{1}{3} + 2 = \frac{7}{3}. \]

إذن
\[ \int_{\gamma_2} F = \frac{7}{3}. \]
\end{exemple}

هذه الحقيقة تُظهر أن التكامل الخطي يعتمد على المسار المتخذ, وأن هذا السلوك ليس عاما كما كان في المثال السابق.

\begin{definition}
لنفترض أن \( F = (f_1, \ldots, f_n) \) هو حقل متجهات على مجموعة مفتوحة \( U \) من \( \mathbb{R}^n \) مع 1-form المرتبطة \( \omega = f_1 \cdot dx_1 + \cdots + f_n \cdot dx_n \). إذا كانت هناك دالة من الفئة \( C^1 \), \( f : U \subset \mathbb{R}^n \to \mathbb{R} \), بحيث أن \( \nabla f = F \) (أو بشكل مكافئ, \( df = \omega \)) على \( U \), فإن حقل المتجهات \( F \) يُقال إنه محافظ و1-form \( \omega \) يُقال إنها دقيقة. الدالة العددية \( f \) تُسمى دالة الجهد لحقل المتجهات المحافظ \( F \).
\end{definition}

تتمتع الحقول المحافظة بسلوك مشابه لذلك في المثال السابق من حيث أن تكاملاتها الخطية مستقلة عن المسار. هذه نتيجة فورية للنتيجة التالية, التي هي تعميم لنظرية التفاضل الأساسية.

\begin{theoreme}[2.5.1]
لنفترض أن \( g : U \subset \mathbb{R}^n \to \mathbb{R} \) دالة من الفئة \( C^1 \) على مجموعة مفتوحة \( U \) وأن \( \gamma : [a, b] \to U \) مسار من الفئة \( C^1 \) من قطعة إلى قطعة. إذن
\[ \int_\gamma \nabla g = \int_\gamma dg = g(\gamma(b)) - g(\gamma(a)). \]
\end{theoreme}

\begin{demonstration}

بما أن \( dg \) هو 1-form المرتبط بحقل المتجهات \( \nabla g \), فإن المعادلة الأولى تتبع من ملاحظتنا المباشرة التي تسبق القسم السابق بالنسبة للمعادلة الثانية, نأخذ التوسع المعتاد لـ 1-form \( dg \),
\[ dg = \sum_{j=1}^n \frac{\partial g}{\partial x_j} \cdot dx_j, \]
ونطبق قاعدة السلسلة,
\[ \sum_{j=1}^n \frac{\partial g}{\partial x_j} (\gamma(t)) \gamma_j'(t) = (g \circ \gamma)'(t), \]
وهي صالحة لجميع \( t \) ما عدا مجموعة منتهية في $[a, b]$, للحصول على
\[
\int_\gamma dg = \int_a^b \sum_{j=1}^n \frac{\partial g}{\partial x_j} (\gamma(t)) \gamma_j'(t) \, dt = \int_a^b (g \circ \gamma)'(t) \, dt. 
\]
لنفرض أن \( P := \{a = t_1 < \cdots < t_k = b\} \) هو تقسيم بحيث أن \( \gamma|_{[t_i, t_{i+1}]} \) من الفئة \( C^1 \) لكل \( 1 \leq i \leq k - 1 \). تعطي نظرية التفاضل الأساسية
\[ \int_\gamma dg = \sum_{i=1}^{k-1} \int_{t_i}^{t_{i+1}} (g \circ \gamma)'(t) \, dt \]

\[ = \sum_{i=1}^{k-1} (g \circ \gamma)(t_{i+1}) - (g \circ \gamma)(t_i) \]

\[ = g(\gamma(b)) - g(\gamma(a)). \]

\end{demonstration}

لتذكيرنا بالمفاهيم المتعلقة بالمجموعات المتصلة والمجموعات المتصلة بالمسار.

\begin{definition}
لنفترض أن \( C \) هو مجموعة جزئية من \( \mathbb{R}^n \).
\begin{enumerate}
    \item[1.]  يُقال إن \( C \) متصلة إذا لم توجد مجموعتان مفتوحتان \( V \) و \( W \) في \( \mathbb{R}^n \) بحيث:
\begin{itemize}
    \item [أ.] \( C \subset V \cup W \);
    \item [ ب.] \( C \cap V \neq \emptyset \) و \( C \cap W \neq \emptyset \);
    \item [ ج.] \( C \cap V \cap W = \emptyset \).
\end{itemize}   
   بمعنى آخر, تكون المجموعة \( C \) متصلة إذا وفقط إذا كانت المجموعة الفرعية من \( C \) التي تكون مفتوحة ومغلقة في آن واحد هي \( C \) نفسها والمجموعة الفارغة \( \emptyset \).
\item[2.]  يُقال إن \( C \) متصلة بالمسار إذا كان لكل \( x, y \in C \), هناك مسار \( \gamma : [a, b] \to C \) بحيث \( \gamma(a) = x \) و \( \gamma(b) = y \). إذا كان يمكن اختيار المسار ليكون مضلعياً, فإن المجموعة تُسمى متصلة مضلعياً.
\end{enumerate}
\end{definition}

الصورة المستمرة لمجموعة متصلة هي متصلة, مما يعني أن كل مجموعة متصلة بالمسار هي متصلة. بالنسبة لمجموعة مفتوحة من \( \mathbb{R}^n \), فإن العكس أيضًا صحيح, وتتماهى المفاهيم الاثنين.

\begin{theoreme}[النظرية 2.5.2]
   
لنفترض أن \( U \) مجموعة مفتوحة متصلة من \( \mathbb{R}^n \). إذن \( U \) متصلة مضلعياً.
\end{theoreme}

\begin{theoreme}[النظرية 2.5.3]
    
لنفترض أن \( F : U \subset \mathbb{R}^n \to \mathbb{R}^n \) حقل متجهات مستمر على مجموعة مفتوحة \( U \). الشروط التالية متكافئة:
\begin{enumerate}
    \item [1.] \( F \) هو حقل متجهات محافظ.
    \item [2.] إذا كان \( \gamma : [a, b] \to U \subset \mathbb{R}^n \) مسار من الفئة \( C^1 \) من قطعة إلى قطعة ومغلق (أي \( \gamma(a) = \gamma(b) \)), إذن
    \[ \int_\gamma F = 0. \]
    \item [3.] إذا كان \( \gamma_1 \) و \( \gamma_2 \) مسارين من الفئة \( C^1 \) من قطعة إلى قطعة ولهما نفس النقطة الابتدائية والنقطة النهائية, إذن
    \[ 
    \int_{\gamma_1} F = \int_{\gamma_2} F. 
    \]
\end{enumerate}
\end{theoreme}

(1) $\Leftarrow$ (2). وفقًا للفرضية, هناك دالة \( f \) من الفئة \( C^1 \) على \( U \), بحيث \( \nabla f = F \). لنفترض أن \( \gamma : [a, b] \to U \) مسار مغلق من الفئة \( C^1 \) من قطعة إلى قطعة. وفقًا للنظرية 2.5.1,
\[ \int_\gamma F = f(\gamma(b)) - f(\gamma(a)) = 0. \]

(2) $\Leftarrow$ (3). إذا كان \( \gamma_1 \) و \( \gamma_2 \) لهما نفس النقطة الابتدائية والنقطة النهائية, إذن \( \gamma := \gamma_1 \cup (-\gamma_2) \) هو مسار مغلق من الفئة \( C^1 \) من قطعة إلى قطعة في \( U \). بالتالي,
\[ \int_{\gamma_1} F - \int_{\gamma_2} F = \int_\gamma F = 0. \]

(3) $\Leftarrow$ (1). لنفترض أولاً أن \( U \) مجموعة مفتوحة متصلة في \( \mathbb{R}^n \). إذن \( U \) متصلة بالمسار ومتصلة مضلعياً. نثبت نقطة \( x_0 \) في \( U \) ونعرف دالة الجهد \( f : U \subset \mathbb{R}^n \to \mathbb{R} \) كما يلي (الشكل 2.5): لكل \( x \in U \) لنفترض أن \( \gamma_x \) هو مسار مضلعي في \( U \) من \( x_0 \) إلى \( x \) ونحدد
\[ f(x) := \int_{\gamma_x} F. \]
تتضمن فرضية (3) أن \( f(x) \) لا تعتمد على المسار المضلع المختار \( \gamma_x \), وبالتالي يكون التعريف غير ملتبس. لإثبات النتيجة, سنبين أن \( f \) هي دالة من الفئة \( C^1 \) والتي يتطابق تدرجها مع حقل المتجهات \( F \).

لنفترض \( x \in U \), يمكننا إيجاد \( \delta > 0 \) بحيث تكون الكرة المفتوحة \( B(x, \delta) \) المتمركزة عند \( x \) ونصف قطرها \( \delta \) محتواة في \( U \). نلاحظ أنه لكل \( 1 \leq j \leq n \) و \( 0 < |t| < \delta \), إذا كان \( \gamma_x \) هو مسار مضلعي في \( U \) من النقطة \( x_0 \) إلى النقطة \( x \), فإن \( \gamma_x \cup [x, x + t e_j] \) هو مسار مضلعي في \( U \) من \( x_0 \) إلى \( x + t e_j \). لذا
\[ f(x + t e_j) - f(x) = \int_{\gamma_x \cup [x, x + t e_j]} F - \int_{\gamma_x} F = \int_{[x, x + t e_j]} F. \]

لنفترض للبساطة أن \( t > 0 \) ونقوم للبارامترة الجزء الخطي بواسطة \( \gamma(s) = x + s e_j \), \( 0 \leq s \leq t \). نكتب أيضًا \( F \) بمصطلحات الدوال المكونة, \( F = (f_1, \ldots, f_n) \). إذن لكل \( 1 \leq j \leq n \),
\[ f(x + t e_j) - f(x) = \int_0^t \langle F(x + s e_j), e_j \rangle \, ds = \int_0^t f_j(x + s e_j) \, ds. \]

يتبع ذلك أن
\[ \frac{f(x + t e_j) - f(x)}{t} = \frac{1}{t} \int_0^t f_j(x + s e_j) \, ds. \]

نحسب هذه القيم على التوالي:
\[ \frac{f(x + t e_j) - f(x)}{t} = \frac{1}{t} \int_0^t f_j(x + s e_j) \, ds \leq \frac{1}{t} \int_0^t |f_j(x + s e_j) - f_j(x)| \, ds. \]

بما أن كل \( f_j \) هو دالة مستمرة, فإن التعبير أعلاه يميل إلى الصفر مع اقتراب \( t \) من الصفر. نتيجة لذلك,
\[ \frac{\partial f}{\partial x_j}(x) = f_j(x). \]

بالتحديد, \( f \) هي دالة من الفئة \( C^1 \) و \( \nabla f = F \) على \( U \). في الحالة العامة حيث \( U \) غير متصلة, يمكن تطبيق الحجة أعلاه على كل مكون متصل من \( U \) من أجل إنشاء دالة جهد مناسبة.
\end{demonstration}
\begin{figure}
    \centering
    \includegraphics[width=0.5\linewidth]{polygonal_path.pdf}
    \caption{}
    \label{fig:enter-label}
\end{figure}

\begin{definition}
يُقال إن المجموعة المفتوحة \( U \subset \mathbb{R}^n \) نجميّة بالنسبة للنقطة \( a \in U \) إذا كان الجزء \([a, x]\) مُحتوى في \( U \) لكل \( x \in U \).
\end{definition}

نذكر أن المثلث الذي يضم الرؤوس \( a, x, y \) هو المجموعة \(\{\alpha a + \beta x + \gamma y : 0 \leq \alpha, \beta, \gamma \quad\text{ و}\quad \alpha + \beta + \gamma = 1 \}\). عن طريق حدوده, نفهم القوس المضلع المغلق $[a, x] \cup  [x, y] \cup  [y, a]$.

\begin{lemma}
لنفترض أن \( U \) نجميّة بالنسبة للنقطة \( a \in U \). إذا كانت \( y \in B(x, R) \subset U \), فإن المثلث (الشكل 2.6) ذو الرؤوس \( a, x, y \) يكون محتوى في \( U \).
\end{lemma}

\begin{figure}
    \centering
    \includegraphics[width=0.5\linewidth]{Triangle_lemma.pdf}
    \caption{}
    \label{fig:enter-label}
\end{figure}

\begin{demonstration}
لنفترض أن \( z = \alpha a + \beta x + \gamma y \) معطى, حيث \( \alpha, \beta, \gamma \geq 0 \) و \(\alpha + \beta + \gamma = 1\). إذا كانت \(\alpha = 1\), فإن \( z = a \in U \). في الحالة الأخرى, نضع
\[ u = \frac{\beta x + \gamma y}{1 - \alpha}. \]
بما أن
\[ \frac{\beta}{1 - \alpha} + \frac{\gamma}{1 - \alpha} = 1, \]
نحصل على أن \( u \in [x, y] \subset B(x, R) \subset U \). أخيرًا, من \( z = \alpha a + (1 - \alpha) u \) نستنتج أن
\[ z \in [a, u] \subset U. \]
\end{demonstration}

\begin{theoreme}
إذا كانت المجموعة المفتوحة \( U \) نجميّة بالنسبة للنقطة \( a \in U \), فإن شروط النظرية 2.5.3 مكافئة لما يلي:
(4) إذا كان \( \gamma \) هو حدود مثلث محتوى في \( U \), إذن
\[ \int_\gamma \omega = 0. \]
\end{theoreme}

\begin{demonstration}
بما أن الشرط (2) يعني بوضوح الشرط (4), يكفي أن نثبت أن الشرط (4) يعني الشرط (1). نفعل ذلك من خلال الحجة أن الدالة \( f \) المعرفة بواسطة
\[ f(x) := \int_{[a, x]} F, \]
هي في الواقع دالة الجهد لحقل المتجهات \( F \). لكل \( x \in U \) نختار \( R_x > 0 \) مع \( B(x, R_x) \subset U \). وفقًا للّمة السابقة, لكل \( 1 \leq j \leq n \) ولكل \( t \in \mathbb{R} \) بحيث \(|t| < R_x \), المثلث ذو الرؤوس \( a, x, و x + t e_j \) يكون محتوى في \( U \). لذا, نستنتج من الشرط (4) أن
\[ f(x + t e_j) - f(x) = \int_{[a, x + t e_j]} F - \int_{[a, x]} F = \int_{[x, x + t e_j]} F. \]

الآن, يستمر الحجة كما في إثبات (3) $\Leftarraw$ (1) في النظرية 2.5.3. 
\end{demonstration}
بالطبع, يمكن تفسير النظرية 2.5.3 أيضًا على أنها وصف لتلك الأشكال التفاضلية من الدرجة الأولى التي تكون دقيقة.

\begin{exemple}
    لنفترض أن
\[ \omega = -\frac{y}{x^2 + y^2} \, dx + \frac{x}{x^2 + y^2} \, dy, \]
وهي 1-form مستمرة في \( U := \mathbb{R}^2 \setminus \{(0, 0)\} \). إذن \( \omega \) ليست دقيقة, نظرًا لأن \( \gamma : [0, 2\pi] \to \mathbb{R}^2 \), \( \gamma(t) := (\cos(t), \sin(t)) \), يحدد مسارًا مغلقًا محتوى في \( U \) لـ
\[ \int_\gamma \omega = 2\pi \neq 0. \]

ومع ذلك, إذا اعتبرنا \( V := \mathbb{R}^2 \setminus \{(0, y) : y \in \mathbb{R}\} \) و \( g : V \to (-\pi, \pi) \) معرّفة بواسطة
\[ g(x, y) := \arctan\left(\frac{y}{x}\right), \]
إذن \( \omega = dg \) على \( V \).
\end{exemple}

وهكذا, فإن حقيقة أن حقل المتجهات محافظ يعتمد ليس فقط على تعبير الحقل بل أيضًا على المنطقة \( U \) التي نتعامل معها. أي أن حقل المتجهات الذي يعترف بدالة جهد في مجموعة مفتوحة معينة قد لا يكون محافظًا في مجموعة أكبر.

\begin{exemple}
(
الحقل الجاذبي محافظ.) 
لنفترض وجود جسيم ذو كتلة \( M \) يقع عند الأصل. قوة الجذب المؤثرة على جسيم ذو كتلة وحدة يقع عند النقطة \( (x, y, z) \in \mathbb{R}^3 \setminus \{(0, 0, 0)\} \) هي
\[ F(x, y, z) = -\frac{GM(x, y, z)}{(x^2 + y^2 + z^2)^{3/2}}, \]
حيث \( G \) هو ثابت الجاذبية. بما أن مقدار القوة هو نفسه في جميع النقاط التي تبعد نفس المسافة عن الأصل, يبدو معقولًا أن نتوقع أن يكون هذا صحيحًا أيضًا بالنسبة لدالة الجهد. وبالتالي, نبحث عن دالة لـ \( r = \sqrt{x^2 + y^2 + z^2} \) التي يكون مشتقها \(-\frac{GM}{r^2}\). مثال على مثل هذه الدالة هو
\[ f(x, y, z) := -\frac{GM}{\sqrt{x^2 + y^2 + z^2}}, \]
ومن السهل التحقق من أن \( \nabla f = F \). أي أن \( f \) هي دالة الجهد للحقل الجاذبي.
\end{exemple}

ينبغي ملاحظة أن ما يسمى بجهد الجاذبية في الفيزياء هو الدالة \( V := -f \). وبالتالي, فإن العمل المنجز بواسطة الحقل لنقل جسيم من النقطة \( A \) إلى النقطة \( B \) يكون مستقلًا عن مسار الجسيم, وقيمته هي الفرق بين الجهود \( V(A) - V(B) \).

\begin{theoreme}
لنفترض أن \( F : U \subset \mathbb{R}^n \to \mathbb{R}^n \) هو حقل متجهات محافظ من الفئة \( C^1 \) على مجموعة مفتوحة \( U \) ونكتب \( F := (f_1, f_2, \cdots, f_n) \). إذن
\[ \frac{\partial f_j}{\partial x_k} (x) = \frac{\partial f_k}{\partial x_j} (x) \]
لكل \( j, k \in \{1, \cdots, n\} \) ولكل \( x \in U \).
\end{theoreme}

\begin{demonstration}
   
وفقًا للفرضية, هناك دالة \( g : U \subset \mathbb{R}^n \to \mathbb{R} \) بحيث \( \nabla g = F \). لذلك, \( f_j = \frac{\partial g}{\partial x_j} \), ويمكننا كتابة
\[ \frac{\partial f_j}{\partial x_k} (x) = \frac{\partial^2 g}{\partial x_k \partial x_j} (x) \]
وبالمثل
\[ \frac{\partial f_k}{\partial x_j} (x) = \frac{\partial^2 g}{\partial x_j \partial x_k} (x) \]
لكل \( x \in U \). بما أن \( F \) هو دالة من الفئة \( C^1 \) على \( U \), فإنه يتبع أن \( g \) من الفئة \( C^2 \), ونظرية Schwarz المتعلقة بتناظر المشتقات الجزئية من الدرجة الثانية تعطينا النتيجة.
\end{demonstration}

على وجه الخصوص, إذا كان \( F = (P, Q) \) حقل متجهات محافظ على \( U \subset \mathbb{R}^2 \) من الفئة \( C^1 \), فإن لكل \( (x, y) \in U \),
\[ \frac{\partial Q}{\partial x} (x, y) = \frac{\partial P}{\partial y} (x, y). \]

يتبع أيضًا من التعريف 1.2.11 والنظرية 2.5.5 أن كل حقل متجهات محافظ \( F = (f_1, f_2, f_3) \) على المجموعة المفتوحة \( U \subset \mathbb{R}^3 \) من الفئة \( C^1 \) يحقق
\[ \text{Curl} \, F (x, y, z) = \left( \frac{\partial f_3}{\partial y} - \frac{\partial f_2}{\partial z}, \frac{\partial f_1}{\partial z} - \frac{\partial f_3}{\partial x}, \frac{\partial f_2}{\partial x} - \frac{\partial f_1}{\partial y} \right) = 0 \]
لكل \( (x, y, z) \in U \).
\\
في المثال 2.5.3 أثبتنا أن حقل المتجهات
\[ F = (P, Q) : U \subset \mathbb{R}^2 \to \mathbb{R}^2 \]
المعرف على \( U = \mathbb{R}^2 \setminus \{(0, 0)\} \) بواسطة
\[ P(x, y) = - \frac{y}{x^2 + y^2}, Q(x, y) = \frac{x}{x^2 + y^2} \]
ليس محافظًا. ومع ذلك,
\[ \frac{\partial Q}{\partial x} (x, y) = \frac{\partial P}{\partial y} (x, y) = \frac{y^2 - x^2}{(x^2 + y^2)^2}. \]

لذا, فإن الشرط في النظرية 2.5.5 ليس كافيًا, بشكل عام, لكي يكون حقل المتجهات محافظًا. ومع ذلك, إذا فرضنا شروطًا هندسية معينة على المجال \( U \), فإن الشرط في النظرية 2.5.5 يكون كافيًا, كما تظهر النظرية التالية. قبل عرض النتيجة, نحتاج إلى تذكر الحقيقة التالية حول مشتق التكامل (يُسمى أيضًا المشتق البارامتري).

\begin{lemma}
    
لنفترض أن \( f : [a, b] \times [c, d] \to \mathbb{R} \) هي دالة من الفئة \( C^1 \) وأن \( g : [a, b] \to \mathbb{R} \) معرّفة بواسطة
\[ g(x) = \int_c^d f(x, t) \, dt. \]
إذن \( g \) هي دالة من الفئة \( C^1 \) على \([a, b]\) و
\[ g'(x) = \int_c^d \frac{\partial f}{\partial x} (x, t) \, dt \]
لكل \( x \in [a, b] \).
\end{lemma}

\begin{theoreme}[مبرهنة بوانكاريه]
    

لنفترض أن \( F : U \subset \mathbb{R}^n \to \mathbb{R}^n \) هو حقل متجهات من الفئة \( C^1 \), \( F := (f_1, f_2, \cdots, f_n) \), على مجموعة مفتوحة \( U \) نجميّة بالنسبة للنقطة \( a \). إذا كان
\[ \frac{\partial f_j}{\partial x_k} (x) = \frac{\partial f_k}{\partial x_j} (x) \]
لكل اختيار للفهارس \( 1 \leq j, k \leq n \) ولكل \( x \in U \), إذن \( F \) هو حقل متجهات محافظ.
\end{theoreme}

\begin{demonstration}
    نعرف \( g : U \subset \mathbb{R}^n \to \mathbb{R} \) بواسطة
\[ g(x) := \sum_{j=1}^n \int_0^1 f_j(a + t(x - a)) \cdot (x_j - a_j) \, dt. \]
يترتب على ذلك أن \( g \) هي دالة من الفئة \( C^1 \) على \( U \) وأن
\[ \frac{\partial g}{\partial x_k} (x) = \sum_{j \neq k} (x_j - a_j) \cdot \int_0^1 \frac{\partial f_j}{\partial x_k} (a + t(x - a)) \cdot t(x_k - a_k) \, dt + \int_0^1 \frac{\partial f_k}{\partial x_k} (a + t(x - a)) \cdot t \, dt + \int_0^1 f_k(a + t(x - a)) \cdot t \, dt \]
لكل \( x \in U \). هذا يثبت أن \( \nabla g = F \) وأن \( F \) هو حقل متجهات محافظ على \( U \). 
\end{demonstration}

\begin{lemma}
لنفترض أن \( F : U \subset \mathbb{R}^3 \to \mathbb{R}^3 \) هو حقل متجهات من الفئة \( C^1 \) على مجموعة مفتوحة نجميّة \( U \). إذن \( F \) محافظ على \( U \) إذا وفقط إذا كان \( \text{Curl} \, F = 0 \).
\end{lemma}

\section{ نظرية غرين}
\begin{theoreme}
لنفترض أن \( K \subset \mathبب{R}^2 \) هي مجموعة مدمجة ذات حدود موجهة إيجابيًا ومعلمة بواسطة \( \gamma \) . لنفترض أن \( \omega = P \، dx + Q \، dy \) هي 1-form من الفئة \( C^1 \) على مجموعة مفتوحة \( U \subset \mathبب{R}^2 \) بحيث \( K \subset U \). إذن
\[ \oint_{\partial K} P \، dx + Q \، dy = \iint_K \left( \frac{\partial Q}{\partial x} - \frac{\partial P}{\partial y} \right) \، d(x، y). \]

باستخدام ترميز حقول المتجهات، يمكن كتابة هذا على النحو التالي:
\[ \oint_{\partial K} (P، Q) \cdot d\mathbf{r} = \iint_K \left( \frac{\partial Q}{\partial x} - \frac{\partial P}{\partial y} \right) \، d(x، y). \]
\end{theoreme}

\begin{figure}
    \centering
    \includegraphics[width=0.5\linewidth]{regoin of type I.pdf}
    \caption{}
    \label{fig:enter-label}
\end{figure}

\begin{definition}
يُقال إن المجموعة المدمجة \( K \subset \mathبب{R}^2 \) هي منطقة من النوع الأول (الشكل 2.7) إذا كان من الممكن وصفها كالتالي:
\[ K = \{ (x, y) \in \mathبب{R}^2 : a \leq x \leq b، f_1(x) \leq y \leq f_2(x) \}، \]
حيث \( f_1، f_2 : [a، b] \to \mathبب{R} \) هما دالتان مستمرتان من الفئة \( C^1 \) جزئيًا، و \( f_1 \leq f_2 \).
\end{definition}

الحدود الموجهة إيجابيًا لـ \( K \) هي اتحاد المسارات \( \gamma = \gamma_1 \cup \gamma_2 \cup (-\gamma_3) \cup (-\gamma_4) \)، حيث:
\[ \gamma_1 : [a، b] \to \mathبب{R}^2، \gamma_1(t) := (t، f_1(t))؛ \]
\[ \gamma_2 : [f_1(b)، f_2(b)] \to \mathبب{R}^2، \gamma_2(t) := (b، t)؛ \]
\[ \gamma_3 : [a، b] \to \mathبب{R}^2، \gamma_3(t) := (t، f_2(t))؛ \]
\[ \gamma_4 : [f_1(a)، f_2(a)] \to \mathبب{R}^2، \gamma_4(t) := (a، t). \]

\begin{lemma}
لنفترض أن \( K \) هي مجموعة مدمجة ومنطقة من النوع الأول وأن \( P \) هي دالة مستمرة على \( K \) وتقبل مشتق جزئي مستمر \( \frac{\partial P}{\partial y} \) في جوار \( K \). إذن
\[ \oint_{\partial K} P \، dx = - \iint_K \frac{\partial P}{\partial y} \، d(x، y). \]
\end{lemma}

\begin{demonstration}
بما أن \( \gamma_1'(t) = (1، f_1'(t))، \gamma_2'(t) = (0، 1)، \gamma_3'(t) = (1، f_2'(t))، \gamma_4'(t) = (0، 1) \) لكل \( t \)، لدينا
\[ \oint_{\gamma} P \، dx = \int_{a}^{b} P(t، f_1(t)) \، dt - \int_{a}^{b} P(t، f_2(t)) \، dt. \]

من ناحية أخرى، كل دالة مستمرة على مجموعة مدمجة \( K \) تكون قابلة للتكامل، ووفقًا لنظرية فوبيني،
\[ \iint_K \frac{\partial P}{\partial y} \، d(x، y) = \int_{a}^{b} \left( P(x، f_2(x)) - P(x، f_1(x)) \right) \، dx. \]

بمقارنة هاتين المعادلتين، لدينا
\[ \oint_{\partial K} P \، dx = - \iint_K \frac{\partial P}{\partial y} \، d(x، y). \]
\end{demonstration}

نص نظرية غرين على أن التكامل الخطي لحقل متجه على محيط منحنى مغلق يعادل تكامل التدفق عبر المساحة المحصورة داخل المنحنى. بشكل رياضي, إذا كان \(C\) منحنى مغلق في المستوى, و \(D\) هي المساحة المحصورة داخل \(C\), فإن:
\[ \oint_C (L \, dx + M \, dy) = \iint_D \left( \frac{\partial M}{\partial x} - \frac{\partial L}{\partial y} \right) dx \, dy. \]

\begin{lemma}
إذا كان \(F : U \subset \mathbb{R}^2 \to \mathbb{R}^2\) حقل متجه من الدرجة \(C^1\) وكان \(C\) منحنى مغلق بسيط بحيث أن \(C = \partial D\) لبعض المنطقة \(D\) في \(U\), فإن:
\[ \oint_C F \cdot dr = \iint_D \left( \frac{\partial N}{\partial x} - \frac{\partial M}{\partial y} \right) dx \, dy, \]
حيث أن \(F = (M, N)\).
\end{lemma}

\begin{demonstration}
نظرية غرين تعطينا مباشرة أن التكامل الخطي للحقل المتجه \(F\) حول المنحنى المغلق \(C\) يساوي التكامل الثنائي لتباعد الحقل المتجه عبر المساحة \(D\) المحصورة داخل \(C\).
\end{demonstration}


على الرغم من الاختلافات بين نظريتي التكامل, وتفضيلنا لنظرية ليبيغ, يجب على القارئ أن يكون على دراية بأنه لأغراضنا, ومع الجهد الكافي, من الممكن أن نُظهر أن معظم الحالات التي نواجهها في النص تناسب إطار تكامل ريمان ومحتوى جوردان, وبالتالي يمكن للقارئ أن يستمر في التفكير والتصور من حيث تكامل ريمان. لكننا لن نسد هذه الفجوة صراحة هنا. كتاب يتناول كلا النظريتين ويمكن مقارنتهما هو كتاب أبوستول.

\begin{definition}
\begin{itemize}
    \item [(i)] تُسمى مجموعة \( R = \Pi_{j=1}^n I_j \) مستطيل \( n \) في \( \mathbb{R}^n \) إذا كان كل \( I_j \) فترة محدودة في \( \mathbb{R} \), أي إذا وجدت الأعداد الحقيقية \( a_j \leq b_j \) بحيث \( (a_j, b_j) \subseteq I_j \subseteq [a_j, b_j] \) لكل \( j = 1, \ldots, n \). في هذه الحالة, يتم تعريف مقياس ليبيغ لـ \( R \) كـ
    \[ m(R) = \Pi_{j=1}^n (b_j - a_j). \]
    \item[(ii)] تُسمى المجموعة الفرعية \( N \) من \( \mathbb{R}^n \) مجموعة عديمة إذا كان لأي \( \epsilon > 0 \), توجد سلسلة من المستطيلات \( n \) $(R_k)$ بحيث \( N \subseteq \cup_{k=1}^{\infty} R_k \) و
    \[ \sum_{k=1}^{\infty} m(R_k) < \epsilon. \]
    \item[(iii)] يُقال أن خاصية \( P(x) \) حيث \( x \in \mathbb{R}^n \) صحيحة تقريبًا في كل مكان إذا وجدت مجموعة عديمة \( N \subset \mathbb{R}^n \) بحيث تكون الخاصية \( P(x) \) صحيحة لكل \( x \in \mathbb{R}^n \setminus N \).
    \item[(iv)] بالنظر إلى مجموعة فرعية \( K \) من \( \mathbb{R}^n \), يتم تعريف الدالة المميزة لـ \( K \), والتي تُرمز بـ \( \chi_K \), كما يلي:
    \[ \chi_K(x) = \begin{cases} 
    1 & \text{إذا } x \in K \\
    0 & \text{إذا } x \notin K 
    \end{cases} \]
    \item[(v)] تُسمى الدالة \( \varphi : \mathbb{R}^n \to \mathbb{R} \) دالة خطوات إذا كانت مزيجًا خطيًا من الدوال المميزة للمستطيلات \( n \), أي إذا
    \[ \varphi = \sum_{k=1}^m c_k \chi_{R_k}, \]
حيث \( c_k \in \mathbb{R} \) و \( R_k \) هو مستطيل \( n \).
    \item[(vi)] تُقال المجموعة الفرعية \( M \subset \mathbb{R}^n \) قابلة للقياس إذا وجدت سلسلة $(\varphi_h)\infty_h=1$ من دوال الخطوات ومجموعة عديمة $N$ من \( \mathbb{R}^n \) بحيث تتقارب سلسلة الأعداد الحقيقية $(\varphi_h(x))$ إلى \( \chi_M(x) \) لكل \( x \in \mathbb{R}^n \setminus N \), أي إذا كانت السلسلة $(\varphi_h)\infty_h=1$ 
تتقارب تقريبًا في كل مكان إلى \( \chi_M \).
\end{itemize}
\end{definition}

يمكن إثبات أنه لأي \(\epsilon > 0\), يمكن تغطية مجموعة عديمة بسلسلة من المكعبات \( (R_k) \) التي هي مستطيلات \( n \) بأطوال جوانب متساوية, بحيث
\[ \sum_{k=1}^{\infty} m(R_k) < \epsilon. \]

ستكون عائلة المجموعات القابلة للقياس ليبيغ في \( \mathbb{R}^n \) مهمة جدًا طوال الوقت, لكننا نحتاج فقط إلى الحقائق الأساسية التالية حول هذه العائلة:

1. كل مجموعة مفتوحة وكل مجموعة مغلقة في \( \mathbb{R}^n \) قابلة للقياس.
2. إذا كانت \( M \) مجموعة قابلة للقياس, فإن \( \mathbb{R}^n \setminus M \) أيضًا قابلة للقياس.
3. إذا كانت \( (M_k) \) عائلة (قابلة للعد) من المجموعات القابلة للقياس, فإن \( \cap_{k=1}^{\infty} M_k \) و \( \cup_{k=1}^{\infty} M_k \) مجموعات قابلة للقياس أيضًا.
4. كل مجموعة عديمة قابلة للقياس ليبيغ.

سنحتاج إلى استخدام نظرية فوبيني, ولذلك سنذكرها. نتبع ترميز سترومبرج [18, نظرية 6.121, ص. 352] ونشير إلى ذلك الكتاب لتعريف تكامل ليبيغ في \( \mathbb{R}^n \). يُرمز إلى فضاء الدوال القابلة للتكامل ليبيغ في \( \mathbb{R}^n \) بـ \( L(\mathbb{R}^n) \). إذا كانت \( A \) مجموعة قابلة للقياس في \( \mathbb{R}^n \), سنقول إن الدالة \( f : A \to \mathbb{R} \) قابلة للتكامل ليبيغ على \( A \) إذا تم تمديدها كصفر خارج \( A \) وتم ترميز هذا التمديد بـ \( f \chi_A \), فإن الدالة \( f \chi_A \) قابلة للتكامل ليبيغ على \( \mathbb{R}^n \). في هذه الحالة, وفقًا للتعريف,
\[ \int_A f := \int_A f(x) \, dx := \int_{\mathbb{R}^n} f \chi_A(x) \, dx. \]

إذا كانت \( A = [a, b] \), نحافظ على الترميز الكلاسيكي ونكتب
\[ \int_{[a,b]} f(x) \, dx = \int_a^b f(x) \, dx. \]

في حالة أن \( f \) تحتوي على متغيرين أو ثلاثة, سنكتب
\[ \int_A f(x, y) \, d(x, y) \text{ أو } \int_A f(x, y, z) \, d(x, y, z). \]

نفضل كتابة \( d(x, y) \) بدلاً من \( dx \, dy \) لتجنب الخلط مع حاصل الضرب الخارجي \( dx \wedge dy \) (انظر الفصل السادس).



**النظرية 2.6.1 (نظرية فوبيني).**
لنفترض أن \( f \in L(\mathbb{R}^{n+p}) \). إذن:

(i) \( f_x(y) = f(x, y) \) ينتمي إلى \( L(\mathbb{R}^p) \) تقريبًا في كل مكان في \( \mathbb{R}^n \).

(ii) \( f_y(x) = f(x, y) \) ينتمي إلى \( L(\mathbb{R}^n) \) تقريبًا في كل مكان في \( \mathbb{R}^p \).

(iii) \( F(x) := \int_{\mathbb{R}^p} f_x(y) \, dy \) معرّف (أي أن التكامل موجود) تقريبًا في كل مكان في \( \mathbb{R}^n \) وهو قابل للتكامل ليبيغ في \( \mathbb{R}^n \). كذلك \( G(y) := \int_{\mathbb{R}^n} f_y(x) \, dx \) معرّف تقريبًا في كل مكان في \( \mathbb{R}^p \), وعلاوة على ذلك, \( G \) ينتمي إلى \( L(\mathbb{R}^p) \).

\[ \int_{\mathbb{R}^n} \left( \int_{\mathbb{R}^p} f(x, y) \, dy \right) \, dx = \int_{\mathbb{R}^{n+p}} f(x, y) \, d(x, y) = \int_{\mathbb{R}^p} \left( \int_{\mathbb{R}^n} f(x, y) \, dx \right) \, dy. \]

متغير لنظرية فوبيني يكون مفيدًا لنا هو التالي. لنفترض أن \( A \subset [a, b] \times [c, d] \) مجموعة قابلة للقياس وأن \( f : A \to \mathbb{R} \) دالة قابلة للتكامل ليبيغ على \( A \). لكل \( x \in [a, b] \), لنعرّف \( A_x = \{ y \in \mathbb{R} : (x, y) \in A \} \). إذن:

\[ \int_A f(x, y) \, d(x, y) = \int_a^b \left( \int_{A_x} f(x, y) \, dy \right) \, dx. \]

المفهوم التالي سيتم مناقشته بشكل أكثر دقة في القسم 8.5.

**التعريف 2.6.2.** لنفترض أن \( K \subset \mathbb{R}^2 \) مجموعة مدمجة بحيث أن حدودها \( \partial K \) تساوي \( \gamma([a, b]) \), حيث \( \gamma : [a, b] \to \mathbb{R}^2 \) مسار مغلق جزئيًا من الفئة \( C^1 \). نقول إن \( \gamma \) موجهة إيجابيًا إذا مرت \( \gamma \) على \( \partial K \) مرة واحدة بحيث أن المنطقة \( K \) دائمًا تقع إلى اليسار (انظر الشكل 2.8). لكي نكون دقيقين, يجب أن نفترض أن المجموعة المدمجة \( K \) تحقق الشرط (8.10), انظر على سبيل المثال كارتان [7, 4.4.1].

**النظرية 2.6.2 (نظرية غرين).** لنفترض أن \( K \subset \mathbb{R}^2 \) مجموعة مدمجة ذات حدود موجهة إيجابيًا يحددها \( \gamma \) (كما في التعريف 2.6.2). لنفترض أن \( \omega = P \, dx + Q \, dy \) شكل تفاضلي من الدرجة 1 من الفئة \( C^1 \) في مجموعة مفتوحة \( U \subset \mathbb{R}^2 \) بحيث \( K \subset U \). إذن:

\[ \oint_{\gamma} (P \, dx + Q \, dy) = \iint_K \left( \frac{\partial Q}{\partial x} - \frac{\partial P}{\partial y} \right) \, d(x, y). \]

باستخدام ترميز الحقول المتجهة, يمكن كتابة ذلك كالتالي:

\[ \oint_{\gamma} (\mathbf{P}, \mathbf{Q}) = \iint_K \left( \frac{\partial Q}{\partial x} - \frac{\partial P}{\partial y} \right) \, d(x, y). \]

\begin{figure}
    \centering
    \includegraphics[width=0.5\linewidth]{regoin of type I.pdf}
    \caption{}
    \label{fig:enter-label}
\end{figure}

قد يشعر القارئ بعدم الارتياح مع التعريف 2.6.2 والنظرية 2.6.2. هذا متوقع ومرحب به. لقد واجهنا للمرة الأولى تباينًا معينًا يتم التغاضي عنه غالبًا أو تجنبه, ولكن يجب علينا معالجته بشكل محدد في هذا الكتاب. يتمتع التعريف 2.6.2 والنظرية 2.6.2 بصياغة بديهية ورياضية دقيقة, وهذه الصياغات ليست سهلة التوفيق. إذا نظرنا إلى نظرية غرين من منظور بديهي (عادةً ما يكون ذلك في سياق حساب التفاضل والتكامل المتجه), فإن الصعوبة الحقيقية لا توجد عمومًا. على سبيل المثال, إذا أردنا تطبيقها على الحالات الناشئة في المشكلات الفيزيائية, فإن ما يعنيه اليسار واليمين يكون واضحًا, ولا نتوقع أي صعوبة في تحديد ما إذا كانت المنطقة \( K \) تقع إلى اليسار. ومع ذلك, من وجهة نظر رياضية بحتة (أي, في سياق التحليل المتجهي), فإن تفسير "اليسار" و"اليمين" ليس واضحًا على الإطلاق, وسيتطلب منا جهدًا كبيرًا لتطوير الأدوات المناسبة للتعامل مع هذا المفهوم. لاحقًا, في الفصول 5 و9, سنقدم تعريفًا دقيقًا للتوجيه ونسخة عامة من نظرية غرين. في الوقت الحالي, سنكتفي بالعمل نحو إثبات النظرية في بعض المناطق المحددة \( K \) التي يمكن تعريف التوجيه فيها بسهولة, والتي يتطابق تعريف التوجيه فيها مع حدسنا الفيزيائي.

**التعريف 2.6.3.** يُقال إن المجموعة المدمجة \( K \subset \mathbb{R}^2 \) منطقة من النوع الأول (الشكل 2.9) إذا كان يمكن وصفها على النحو التالي:
\[ K = \{ (x, y) \in \mathbb{R}^2 : a \leq x \leq b, f_1(x) \leq y \leq f_2(x) \}, \]
حيث \( f_1 \) و \( f_2 \) دوال جزئية من الفئة \( C^1 \), و \( f_1 \leq f_2 \).

الحدود الموجهة إيجابيًا لـ \( K \) هي اتحاد المسارات \( \gamma = \gamma_1 \cup \gamma_2 \cup (-\gamma_3) \cup (-\gamma_4) \), حيث:
\[ \gamma_1 : [a, b] \to \mathbb{R}^2, \gamma_1(t) := (t, f_1(t))؛ \]
\[ \gamma_2 : [f_1(b), f_2(b)] \to \mathbb{R}^2, \gamma_2(t) := (b, t)؛ \]
\[ \gamma_3 : [a, b] \to \mathbb{R}^2, \gamma_3(t) := (t, f_2(t))؛ \]
\[ \gamma_4 : [f_1(a), f_2(a)] \to \mathbb{R}^2, \gamma_4(t) := (a, t). \]

**اللمّة 2.6.1.** لنفترض أن \( K \) مجموعة مدمجة من النوع الأول وأن \( P \) دالة مستمرة على \( K \) تقبل مشتقة جزئية مستمرة \( \frac{\partial P}{\partial y} \) في جوار \( K \). إذن:

\[ \oint_{\gamma} P \, dx = -\iint_K \frac{\partial P}{\partial y} \, d(x, y). \]

الدليل: بما أن \( \gamma_1'(t) = \gamma_1'(t) = (0, 1) \), \( \gamma_3'(t) = (1, f_3'(t)) \), و \( \gamma_3'(t) = (1, f_3'(t)) \) لكل \( t \), لدينا:

\[ 
\oint_{\gamma} P \, dx = \int_a^b P(t, f_1(t)) \, dt - \int_a^b P(t, f_2(t)) \, dt.
\]

من ناحية أخرى, كل دالة مستمرة على مجموعة مدمجة \( K \) هي قابلة للتكامل, ووفقًا لنظرية فوبيني:
\[ \iint_K \frac{\partial P}{\partial y} \, d(x, y) = \int_a^b \left( \int_{f_1(x)}^{f_2(x)} \frac{\partial P}{\partial y} (x, y) \, dy \right) dx = \int_a^b (P(x, f_2(x)) - P(x, f_1(x))) \, dx. \]

بمقارنة هاتين المعادلتين, نحصل على:

\[ \oint_{\gamma} P \, dx = -\iint_K \frac{\partial P}{\partial y} \, d(x, y). \]

**التعريف 2.6.4.** يُقال إن المجموعة المدمجة \( K \subset \mathbb{R}^2 \) منطقة من النوع الثاني (الشكل 2.10) إذا كان يمكن وصفها على النحو التالي:
\[ K = \{ (x, y) : c \leq y \leq d؛ g_1(y) \leq x \leq g_2(y) \}, \]
حيث \( g_1 \) و \( g_2 \) دوال جزئية من الفئة \( C^1 \), و \( g_1 \leq g_2 \).

الحدود الموجهة إيجابيًا لـ \( K \) هي اتحاد المسارات \( \gamma = (-\gamma_1) \cup \gamma_2 \cup \gamma_3 \cup (-\gamma_4) \), حيث:
\[ \gamma_1 : [c, d] \to \mathbb{R}^2, \gamma_1(t) := (g_1(t), t)؛ \]
\[ \gamma_2 : [g_1(c), g_2(c)] \to \mathbb{R}^2, \gamma_2(t) := (t, c)؛ \]
\[ \gamma_3 : [c, d] \to \mathbb{R}^2, \gamma_3(t) := (g_2(t), t)؛ \]
\[ \gamma_4 : [g_1(d), g_2(d)] \to \mathbb{R}^2, \gamma_4(t) := (t, d). \]


**اللمّة 2.6.2.** لنفترض أن \( K \) مجموعة مدمجة وهي منطقة من النوع الثاني, ولنفرض أن \( Q \) دالة مستمرة على \( K \) وتقبل مشتقة جزئية مستمرة \( \frac{\partial Q}{\partial x} \) في جوار \( K \). إذن:

\[ \oint_{\gamma} Q \, dy = \iint_K \frac{\partial Q}{\partial x} \, d(x, y). \]

**النظرية 2.6.3.** لنفترض أن \( K \) منطقة من النوع الأول أو منطقة من النوع الثاني, ولنفترض أن \( \omega = P \, dx + Q \, dy \) شكل تفاضلي من الدرجة 1 من الفئة \( C^1 \) على بعض المستطيلات المفتوحة التي تحتوي على \( K \). إذن:

\[ \oint_{\gamma} P \, dx + Q \, dy = \iint_K \left( \frac{\partial Q}{\partial x} - \frac{\partial P}{\partial y} \right) \, d(x, y). \]

الدليل: نختار \( a \), \( b \), \( c \), و \( d \) بحيث تكون \( K \) محتواة في المستطيل \( R = [a, b] \times [c, d] \) ونفترض أن الشكل التفاضلي \( \omega \) معرف وهو من الفئة \( C^1 \) على المستطيل المفتوح \( T \) الذي يحتوي على \( R \). سنعرض الدليل في حالة أن \( K \) منطقة من النوع الأول. في حالة أن \( K \) منطقة من النوع الثاني, فإن حجة مشابهة تؤدي إلى نفس النتيجة.

لقد أثبتنا بالفعل أن:

\[ \oint_{\gamma} P \, dx = -\iint_K \frac{\partial P}{\partial y} \, d(x, y), \]

ولذلك نحتاج فقط إلى الحصول على:

\[ \oint_{\gamma} Q \, dy = \iint_K \frac{\partial Q}{\partial x} \, d(x, y). \]

لاحظ أننا لا يمكننا تطبيق اللمّة السابقة, لأن \( K \) ليست بالضرورة منطقة من النوع الثاني. نتابع كما يلي.

لكل \( (x, y) \in T \), نعرّف \( V(x, y) := \int_c^y Q(x, t) \, dt \). الدالة الناتجة \( V \) لها خاصية أن:

\[ dV(x, y) = F(x, y) \, dx + Q(x, y) \, dy, \]

حيث:

\[ F(x, y) = \int_c^y \frac{\partial Q}{\partial x} (x, t) \, dt. \]

تم الحصول على التعبير عن \( F \) بعد تطبيق الاقتراح 2.5.1 (انظر أيضًا أبوستول [1, نظرية 10.39]). وفقًا للنظرية 2.5.1, لدينا:

\[ dV = 0, \]

لذلك:

\[ \oint_{\gamma} Q(x, y) \, dy = -\oint_{\gamma} F(x, y) \, dx. \]

علاوة على ذلك, \( K \) هي منطقة من النوع الأول, وفي جوار \( K \):

\[ \frac{\partial F}{\partial y} (x, y) = \frac{\partial Q}{\partial x} (x, y). \]

تطبيق اللمّة 2.6.1 على \( F(x, y) \, dx \) يعطينا:

لأن لكل \( x \) ثابت, تكون الدالة \( y \to F(x, y) \) بدائية لـ \( y \to \frac{\partial Q}{\partial x} (x, y). \)

أخيرًا:

\[ \oint_{\gamma} Q \, dy = \iint_K \frac{\partial Q}{\partial x} \, d(x, y), \]

و:

\[ \oint_{\gamma} P \, dx + Q \, dy = \iint_K \left( \frac{\partial Q}{\partial x} - \frac{\partial P}{\partial y} \right) \, d(x, y). \]


**التعريف 2.6.5.** تُقال أن المجموعة المدمجة \( K \) هي منطقة بسيطة إذا كانت \( K \) منطقة من النوع الأول وأيضًا منطقة من النوع الثاني.

يمكن تخفيف الفرضية الخاصة بنظرية غرين على المناطق البسيطة إلى حد ما, حيث يكفي أن يكون الشكل التفاضلي من الدرجة الأولى معرفًا في جوار \( K \). قبل تقديم الدليل, دعونا نحدد اتفاقية الترميز التي تكون صالحة بناءً على الحقيقة التالية (انظر الاقتراح 2.7.2). لنفترض أن \( K \) منطقة بسيطة و \( \gamma_1 : [a, b] \to \mathbb{R}^2 \), \( \gamma_2 : [c, d] \to \mathbb{R}^2 \) هما مساران بسيطان, مغلقان, وقطعيا ناعمان مع
\[ \gamma_1([a, b]) = \gamma_2([c, d]) = \partial K \]
وبحيث أن \( \gamma_1 \) و \( \gamma_2 \) ينتجان نفس التوجيه على الحدود \( \partial K \). إذن وفقًا للاقتراح 2.7.2,
\[ \oint_{\gamma_1} \omega = \oint_{\gamma_2} \omega. \]

بالتالي, من المنطقي كتابة
\[ \oint_{\partial K} \omega \]
إذا فسرنا التكامل على أنه تم على طول أي مسار بسيط, مغلق, وقطعيًا \( C^1 \) يعطي تبارجعًا لـ \( \partial K \) بحيث تكون المنطقة \( K \) دائمًا على اليسار.

**النظرية 2.6.4.** لنفترض أن \( K \subset \mathbb{R}^2 \) منطقة بسيطة وأن \( \omega = P \, dx + Q \, dy \) شكل تفاضلي من الدرجة الأولى من الفئة \( C^1 \) على جوار مفتوح لـ \( K \). إذن:

\[
\oint_{\partial K} \omega = \iint_K \left( \frac{\partial Q}{\partial x} - \frac{\partial P}{\partial y} \right) \, d(x, y).
\]

الدليل: بما أن \( K \) منطقة من النوع الأول, لدينا:

\[ 
\oint_{\partial K} P \, dx = -\iint_K \frac{\partial P}{\partial y} \, d(x, y), 
\]

وبما أنها أيضًا منطقة من النوع الثاني, نحصل على:
\[ 
\oint_{\partial K} Q \, dy = \iint_K \frac{\partial Q}{\partial x} \, d(x, y). 
\]

يكفي جمع هاتين الهويتين.

تسمح لنا صيغة غرين بتقديم دليل مباشر لشرط الكفاية لحقل المتجه ليكون محافظًا (النظرية 2.5.6) على مجموعة مفتوحة ونجومية في المستوى.


**النظرية 2.6.5 (لمّة بوانكاريه).** لنفترض أن \( F = (P, Q) \) حقل متجه من الفئة \( C^1 \) على المجموعة المفتوحة والنجومية \( U \subset \mathbb{R}^2 \) ونفترض أن:

\[ \frac{\partial Q}{\partial x} = \frac{\partial P}{\partial y} \]
لكل \( (x, y) \in U \). إذن \( F \) هو حقل متجه محافظ.

الدليل: لنفترض أن \( \omega = P \, dx + Q \, dy \) هو الشكل التفاضلي المرتبط بحقل المتجه \( F \). يكفي إثبات أن:


\[ \oint_{\partial K} \omega = 0 \]
كلما كانت \( \partial K \) حدود مثلث ما \( K \) محتواة في \( U \). بما أن \( K \) منطقة بسيطة, فإنه يتبع من نظرية غرين أن:

\[ \oint_{\partial K} (P \, dx + Q \, dy) = \iint_K \left( \frac{\partial Q}{\partial x} - \frac{\partial P}{\partial y} \right) \, d(x, y) = 0. \]

ستُعرض نسخة عامة من لمّة بوانكاريه على \( \mathbb{R}^n \) في الفصل السادس (انظر القسم 6.6) بعد أن نكمل دراستنا للأشكال التفاضلية.
\newpage

\section{التمارين}
\subsection{}
احسب طول المسار
\[ \gamma(t) = (e^t \cos(t), e^t \sin(t)), \]
للفترة \( 0 \leq t \leq \pi \).
\subsection{}
احصل على تبارجع, عكس اتجاه عقارب الساعة, للإهليلج
\[ \frac{x^2}{a^2} + \frac{y^2}{b^2} = 1 \] 
(حيث \( a, b > 0 \)).

\subsection{}
تكامل حقل المتجه
\[ F(x, y) = (y^2, -2xy) \]
على طول المثلث ذو الرؤوس \((0, 0), (1, 0), (0, 1)\), مع الاتجاه عكس اتجاه عقارب الساعة.

\subsection{}
(1) جد مسارًا \( \gamma \) يكون مساره هو تقاطع الأسطوانة \( x^2 + y^2 = 1 \) مع المستوي \( x + y + z = 1 \) ومع الخصائص الإضافية أن النقطة الابتدائية (والنقطة النهائية) هي \((0, -1, 2)\) والإسقاط على مستوى \( xy \) هو باتجاه عكس عقارب الساعة.
(2) احسب:
\[ \int_{\gamma} (x \, dy + y \, dz - x \, dz). \]

\subsection{}
لنفترض أن \( \gamma \) هو مسار بسيط يكون مساره هو تقاطع المستويات الإحداثية مع جزء الكرة الوحدة في الربع الأول, موجهًا وفقًا للتسلسل \((1, 0, 0), (0, 1, 0), (0, 0, 1), (1, 0, 0)\). احسب:
\[ \int_{\gamma} (z \, dx + x \, dy + y \, dz). \]
\subsection{}
جد مسارًا يكون مساره هو تقاطع نصف الكرة العلوي ذي نصف القطر \( 2a \) مع الأسطوانة
\[ x^2 + y^2 + z^2 = 4a^2, x^2 + (y - a)^2 = a^2. \]

\subsection{}
حدد ما إذا كان حقل المتجه
\[ F : \mathbb{R}^2 \to \mathbb{R}^2, F(x, y) = (x^3y, x) \]
محافظًا.
\subsection{}
جد دالة جهد لحقل المتجه
\[ F : \mathbb{R}^2 \to \mathbb{R}^2, F(x, y) = (2xy, x^2 - y^2). \]
\subsection{}
لحقل المتجه
\[ F : \mathbb{R}^3 \to \mathbb{R}^3, F(x, y, z) = (2xy, x^2 + z^2, 2zy), \]
(1) أثبت أن \( \nabla \times F = 0 \);
(2) جد دالة جهد لـ \( F \).
\subsection{}
(1) أثبت أن حقل المتجه
\[
F(x, y) = \left(e^x(\sin(x + y) + \cos(x + y)) + 1, e^x \cos(x + y) \right)
\]
محافظ على \( \mathbb{R}^2 \) واجد دالة جهد.
\\
(2) احسب التكامل الخطي
\[ \int_{\gamma} F, \]
حيث
\[ \gamma : [0, \pi] \to \mathbb{R}^2;\, \gamma(t) = (\sin(\pi e^{\sin(t)})), \cos^5(t)). \]
\subsection{}
احسب
\[ 
\int_{\gamma} x \, dx + y \, dy + z \, dz, 
\]
حيث:
\[ \gamma(t) = (\cos^4(t), \sin^2(t) + \cos^3(t), t), 0 \leq t \leq \pi. \]

\subsection{}
للحقل المتجه
\[ F = (f_1, f_2) : \mathbb{R}^2 \setminus \{(0, 0)\} \to \mathbb{R}^2, \]
المعرف بـ:
\[ f_1(x, y) = -\frac{y}{x^2 + y^2}, f_2(x, y) = \frac{x}{x^2 + y^2}, \]

(1) أثبت أن
\[ \frac{\partial f_2}{\partial x}(x, y) = \frac{\partial f_1}{\partial y}(x, y) \]
لكل \( (x, y) \in \mathbb{R}^2 \setminus \{(0, 0)\).

(2) لنفترض أن \( \gamma \) هي الدائرة الوحدة موجهة بعكس اتجاه عقارب الساعة. أثبت أن
\[ \oint_{\gamma} F = 2\pi. \]
هل هذه الحقيقة تتناقض مع لمّة بوانكاريه؟

(3) ناقش ما إذا كان هذا البيان صحيحًا: لكل مسار مغلق وقطعيًا من الفئة \( C^1 \) \( \alpha : [a, b] \to \mathbb{R}^2 \setminus \{(0, 0)\} \) بحيث \( \alpha_1(t) \geq 0 \) لكل \( t \in [a, b] \),
\[ \oint_{\alpha} F = 0. \]

(4) احسب \( \oint_{\gamma} F \), حيث
\[ \gamma(t) = (\cos(t), \sin^7(t)), -\frac{\pi}{2} \leq t \leq \frac{\pi}{2}. \]
\subsection{}
لنفرض أن \( \gamma(t) = (\cos(t), \sin(t)) \) معطاة ولننظر في المسارين
\[ \alpha := \gamma|_{[-\pi, \pi]} \text{ و } \beta := \gamma|_{[0, 2\pi]}. \]

(1) هل \( \alpha \) و \( \beta \) مساران مكافئان؟

(2) برر لماذا
\[ \oint_{\alpha} F = \oint_{\beta} F \]
لكل حقل متجه مستمر \( F : U \subset \mathbb{R}^2 \to \mathbb{R}^2 \) معرف على الدائرة الوحدة.
\subsection{}
لنفترض أن \( \gamma \) هو المسار الذي يكون مساره اتحاد رسم الدالة \( y = x^3 \) من النقطة (0, 0) إلى النقطة (1, 1) والقطعة من النقطة (1, 1) إلى النقطة (0, 0). باستخدام نظرية غرين, احسب:
\[ \int_{\gamma} (x^2 + y^2) \, dx + (2xy + x^2) \, dy. \]
\subsection{}
استخدم نظرية غرين لحساب التكامل الخطي:
\[ 
\int_{\gamma} (x^2 + 3x^2y^2) \, dx + (2x^3y + x^2) \, dy, 
\]
حيث \( \gamma \) هو الحد الفاصل للمنطقة الواقعة بين رسوم \( y = 0 \) و \( y = 4 - x^2 \), موجهة بعكس اتجاه عقارب الساعة.
\subsection{}
لنفترض أن \( K \) منطقة من النوع الأول أو النوع الثاني وأن \( \gamma \) مسار يكون مساره هو الحد الفاصل لـ \( K \) موجهًا بعكس اتجاه عقارب الساعة. إذن:
\[ \text{area}(K) = \frac{1}{2} \oint_{\gamma} (-y \, dx + x \, dy) = \frac{1}{2} \oint_{\gamma} -y \, dx = \frac{1}{2} \oint_{\gamma} x \, dy. \]
\subsection{}
احسب المساحة المحاطة بالدوارة \( \gamma : [0, 2\pi] \to \mathbb{R}^2 \),
\[ \gamma(t) = (at - a\sin(t), a - a\cos(t)) \]
(حيث \( a > 0 \)) ومحور \( x \).
\subsection{}
احسب المساحة المحاطة بالمحورين الإحداثيين والمسار
\[ \gamma(t) = (\sin^4(t), \cos^4(t)), 0 \leq t \leq \pi. \]
\subsection{}
احسب المساحة المحصورة بين الدائرتين
\[ C_1: (x, y) \in \mathbb{R}^2 : x^2 + y^2 = a^2 \]
و
\[ C_2: (x, y) \in \mathbb{R}^2 : x^2 + y^2 = 2ax  ( a > 0 ). \]
