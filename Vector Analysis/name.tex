
\section{تكامل الأشكال التفاضلية}

لنعتبر شكل تفاضلي من الدرجة \(1\) في المجال \(U\), أي دالة من الشكل \(\omega = f_1 \, dx_1 + f_2 \, dx_2 + \cdots + f_n \, dx_n\), حيث أن \(f_i : U \to \mathbb{R}\) دوال مستمرة. إذا كانت \(\alpha : [a, b] \to U\) مسارًا منتظمًا, فإننا نعرف تكامل \(\omega\) على \(\alpha\) كما يلي:
\[ \int_\alpha \omega = \int_a^b (f_1(\alpha(t)) \dot{\alpha}_1(t) + f_2(\alpha(t)) \dot{\alpha}_2(t) + \cdots + f_n(\alpha(t)) \dot{\alpha}_n(t)) \, dt. \]

الذي يمكن كتابته بشكل مختصر كـ:
\[ \int_\alpha \omega = \int_a^b \omega(\alpha(t), \dot{\alpha}(t)) \, dt. \]




تُظهر نظرية غرين أن هذا الاستدلال صحيح في بعض الحالات. حتى هذه النقطة, تعاملنا فقط مع التكامل للدوال المستمرة على الفترة \([a, b]\), ولذلك احتجنا فقط إلى الخصائص الأساسية لتكامل ريمان. من الآن فصاعدًا, سنحتاج إلى أدوات من نظرية التكامل في عدة متغيرات. في هذا السياق, يجب علينا الاختيار بين خيارين: تكامل ريمان وتكامل ليبيغ. تكامل ريمان على مستطيل في \(\mathbb{R}^n\), وامتداده إلى المجموعات المحدودة ذات المحتوى الجورداني, والتي تُسمى عادةً المجموعات القابلة للقياس الجورداني, هو الأكثر بديهية بشكل كبير. من ناحية أخرى, تكامل ليبيغ على \(\mathbb{R}^n\), أو تحديدًا, على مجموعة قابلة للقياس في \(\mathbb{R}^n\), هو النظرية الأكثر قوة, حتى وإن كانت أقل بديهية. لقد قررنا استخدام تكامل ليبيغ. لماذا؟ النقطة الرئيسية هي أن أي مجموعة مفتوحة أو مغلقة في \(\mathbb{R}^n\) قابلة للقياس ليبيغ, ولكنها عمومًا ليست قابلة للقياس جورداني. علاوة على ذلك, كل دالة مستمرة على مجموعة مدمجة هي قابلة للتكامل ليبيغ, واثنين من أهم نظريات حساب التكامل, وهما نظرية فوبيني ونظرية تغيير المتغير, ينطبقان على نطاق أوسع بكثير في تكامل ليبيغ مقارنة بتكامل ريمان.
\section{ التمارين}

**التمرين 2.8.1.** احسب التكامل الخطي للحقل المتجه \(F(x, y) = (x^2 - y^2, 2xy)\) على المسار \( \alpha(t) = (t, t^2) \) للفترة \(0 \leq t \leq 1\).

**التمرين 2.8.2.** أثبت أن الحقل المتجه \(F(x, y) = (-y, x)\) هو حقل محافظ واحسب دالة الجهد المقابلة له.

**التمرين 2.8.3.** احسب التكامل الخطي للحقل المتجه \(F(x, y) = (2x, 3y^2)\) على المسار المغلق \( \alpha(t) = (\cos t, \sin t) \) للفترة \(0 \leq t \leq 2\pi\).

**التمرين 2.8.4.** باستخدام نظرية غرين, احسب التكامل الخطي للحقل المتجه \(F(x, y) = (y, x)\) حول المربع ذي الزوايا عند \((1, 1)\), \((1, -1)\), \((-1, -1)\), و\((-1, 1)\).



\section{ ملحق: تعليقات على التغيير في المعلمة}

أحيانًا يكون من الضروري تغيير المعلمة للتكاملات الخطية. على سبيل المثال, إذا كان لدينا مسار \(\alpha : [a, b] \to \mathbb{R}^n\) ودالة قابلة للتفاضل \(\beta : [c, d] \to [a, b]\) بحيث \(\beta(c) = a\) و \(\beta(d) = b\), فإن المسار \(\gamma : [c, d] \to \mathbb{R}^n\) المعرف بواسطة \(\gamma(t) = \alpha(\beta(t))\) سيسمح لنا بإعادة صياغة التكامل كالتالي:
\[ \int_\gamma F \cdot d\gamma = \int_c^d F(\alpha(\beta(t))) \cdot \dot{\alpha}(\beta(t)) \beta'(t) \, dt. \]


