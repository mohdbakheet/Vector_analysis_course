\chapter*{ مقدمة}
 تحليل المتجهات جزء أساسي من منهاج الرياضيات المتقدمة الذي يهدف إلى تقديم المفاهيم والأدوات الرياضية المستخدمة في تحليل الحقول المتجهية ودراسة التكاملات المتعددة. هذه الملاحظات المحاضرة تهدف إلى توفير فهم عميق وشامل للمواضيع المختلفة التي سيتم تناولها خلال الكورس، مع التركيز على التطبيقات العملية والنظرية.

\section*{ ما هو تحليل المتجهات؟}

تحليل المتجهات هو فرع من الرياضيات يهتم بدراسة الكميات المتجهة وتطبيقاتها. الكميات المتجهة، مثل القوة والسرعة، تتميز بامتلاكها لكل من الحجم والاتجاه. يتم استخدام تحليل المتجهات لتحليل الحقول المتجهية، والتي تمثل كيفية تغير الكميات المتجهة في الفضاء.

\section*{ الأهداف التعليمية }
\begin{enumerate}
    \item تطوير فهم عميق للمفاهيم الأساسية في تحليل المتجهات، مثل المتجهات، والنقاط، والخطوط، والسطوح.
    \item  تطبيق العمليات الرياضية الأساسية على المتجهات، بما في ذلك الجمع، والطرح، والضرب الاتجاهي، والضرب القياسي.
    \item . دراسة التكاملات الخطية وتطبيقاتها في الفيزياء والهندسة، مثل حساب العمل المنجز بواسطة قوة متغيرة.
    \item  فهم واستخدام النظريات الأساسية مثل نظرية غرين ونظرية ستوكس ونظرية التباعد (غاوس).
    \item  تطبيق المفاهيم الرياضية لحل المشاكل الواقعية في مجالات الفيزياء والهندسة.
\end{enumerate}
\section*{ محتويات الملاحظات}

تتضمن ملاحظات المحاضرة الموضوعات التالية:
\begin{enumerate}
    \item مقدمة في المتجهات والمتجهات المكانية: تعريف المتجهات، العمليات على المتجهات، والتمثيلات المكانية.
    \item  الحساب التفاضلي للمتجهات: دراسة التدرجات، والتباعدات، والتدويرات، وتطبيقاتها.
    \item التكاملات المتجهية: التكاملات الخطية، وتطبيقاتها في حساب العمل والطاقة.
    \item . نظريات الحقول المتجهية: نظريات غرين وستوكس وغاوس، وتطبيقاتها في الفيزياء والهندسة.

\section*{ كيفية استخدام هذه الملاحظات}

تم تصميم هذه الملاحظات لتكون مرجعًا دراسيًا يساعد الطلاب في فهم المواضيع بشكل أعمق والتحضير للامتحانات في ظل هذه الظروف الاستثنائية. ينصح الطلاب بقراءة الملاحظات بانتظام، وحل التمارين المرفقة، ومراجعة المفاهيم الأساسية بشكل مستمر.

نأمل أن تجدوا هذه الملاحظات مفيدة ونتطلع إلى رحلة تعليمية مثمرة.
\\
\begin{center}
.اعاننا الله واياكم عل اتمام المقرر بصورة مرضية
    
\end{center}
